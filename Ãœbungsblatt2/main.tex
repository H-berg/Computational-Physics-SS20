\documentclass[titlepage=firstiscover, captions=tableheading, bibliography=totoc]{scrartcl}
\usepackage[autostyle=true,german=quotes]{csquotes}
\usepackage{scrhack}
\usepackage{enumitem}
\usepackage{caption}
\usepackage[aux]{rerunfilecheck}
\usepackage{subcaption}
\usepackage{fontspec}
\usepackage[dvips]{graphicx}
\usepackage{floatflt,epsfig}

\usepackage{polyglossia}
\setmainlanguage{german}

\usepackage[unicode]{hyperref}
\usepackage{bookmark}
\title{Computational Physics}
\subtitle{Übungsblatt 7}
\author{
Miriam Simm\\
\texorpdfstring{\href{mailto:miriam.simm@tu-dortmund.de}{miriam.simm@tu-dortmund.de}\and}{,}
Katrin Bolsmann\\
\texorpdfstring{\href{mailto:katrin.bolsmann@tu-dortmund.de}{katrin.bolsmann@tu-dortmund.de}}{,}\\
\\
Mario Alex Hollberg\\
\texorpdfstring{\href{mailto:mario-alex.hollberg@tu-dortmund.de}{mario-alex.hollberg@tu-dortmund.de}}{}
}
\date{Abgabe: 15. Mai 2020}
\usepackage{amsmath}
\usepackage{amssymb}
\usepackage{mathtools}
\usepackage[
    math-style=ISO,
    bold-style=ISO,
    sans-style=italic,
    nabla=upright,
    partial=upright,
]{unicode-math}

\setmathfont{Latin Modern Math}

\usepackage[
  locale=DE,
  separate-uncertainty=true,
  per-mode=symbol-or-fraction,
]{siunitx}

\usepackage{multicol}
\setlength{\columnsep}{1pt} %space between columns

\usepackage{booktabs}
\usepackage[x11names, table]{xcolor}
\usepackage{graphicx}
\usepackage{grffile}
\usepackage{xfrac}
\usepackage{xcolor}

\usepackage{float}
\floatplacement{figure}{h}
\floatplacement{table}{h}
\usepackage[
  section,
  below,
]{placeins}

\usepackage{expl3}
\usepackage{xparse}
\ExplSyntaxOn
\NewDocumentCommand \E {} {\symup{e}}
\ExplSyntaxOff

% Literaturverzeichnis
\usepackage[
  backend=biber,
]{biblatex}
% Quellendatenbank
\addbibresource{literatur.bib}

\usepackage[
  version=4,
  math-greek=default,
  text-greek=default,
]{mhchem}


\raggedcolumns


\begin{document}

\maketitle

\section*{Aufgabe 1: Singulärwertzerlegung}
\subsection*{a)}
Das Bild wird mittels der Datei $\texttt{service.cpp}$ eingelesen und
zunächst durch Transposition gedreht, sodass der Mandrill einem in die Augen
sieht (siehe Abbildung \ref{fig:1a}). Im nächsten Schritt wird mithilfe von
$\texttt{Eigen::BDCSVD}$ eine Singulärwertzerlegung durchgeführt.

\begin{figure}[h]
    \centering
    \includegraphics[width=0.8\textwidth]{bild_rang_0.pdf}
    \caption{Transponierte Bild-Daten}
    \label{fig:1a}
\end{figure}

\subsection*{b)}
In diesem Aufgabenteil wird eine Rang-$k$-Approximation gemäß der Vorlesung,
für $k$-Werte von  10, 20 und 50, durchgeführt und im folgenden als Heatmap mit
Graustufen-Farbskala dargestellt.

\begin{figure}[!htb]
\minipage{0.33\textwidth}
\includegraphics[width=\textwidth]{bild_rang_10.pdf}
\caption*{k=10}
\label{fig:1b_10}
\endminipage\hfill
\minipage{0.33\textwidth}
\includegraphics[width=\textwidth]{bild_rang_20.pdf}
\caption*{k=20}
\label{fig:1b_20}
\endminipage\hfill
\minipage{0.33\textwidth}%
\includegraphics[width=\textwidth]{bild_rang_50.pdf}
\caption*{k=50}
\label{fig:1b_50}
\endminipage
\caption{Singulärwertzerlegte Bild-Daten nach Rang-k-Approximation}
\end{figure}

\begin{figure}[h]
    \centering
    \includegraphics[width=0.8\textwidth]{bild_rang_0.pdf}
    \caption{Singulärwertzerlegte Bild-Daten nach Rang-k=10-Approximation}
    \label{fig:1b_10}
\end{figure}

\begin{figure}[h]
    \centering
    \includegraphics[width=0.8\textwidth]{bild_rang_20.pdf}
    \caption{Singulärwertzerlegte Bild-Daten nach Rang-k=20-Approximation}
    \label{fig:1b_20}
\end{figure}

\begin{figure}[h]
    \centering
    \includegraphics[width=0.8\textwidth]{bild_rang_50.pdf}
    \caption{Singulärwertzerlegte Bild-Daten nach Rang-k=50-Approximation}
    \label{fig:1b_50}
\end{figure}

\section*{Aufgabe 2: Profiling zur Untersuchung eines Algorithmus}
In dieser Aufgabe soll ein zufälliges Lineares Gleichungssystem, aus einer quadratischen Matrix $M$ der Dimension $N$ und N-dimensionalen Vektoren $x$ und $b$

\begin{equation*}
  M\, x\, =\, b
\end{equation*}

\noindent
mit einer LU-Zerlegung gelöst werden. Dabei soll die Laufzeit der einzelnen Arbeitsschritte

\begin{itemize}
  \item[1)] Erstellen einer zufälligen N-dimensionalen Matrix $M$
  \item[2)] Durchführung der LU-Zerlegung
  \item[3)] Lösen des Gleichungssystems mit LU-Zerlegung
\end{itemize}

\noindent
für verschiedene Dimensionen $N$ verglichen werden.

\subsection*{b) Plotten der Laufzeiten}
In diesem Aufgabenteil wurde N logarithmisch vergrößert und die Laufzeiten gemessen. Diese sind in Plot \ref{fig:plot2b} doppellogarithmisch aufgetragen. Zusätzlich wurde auch die Gesamtlaufzeit des Algorithmus berechnet.

\FloatBarrier
\begin{figure}[h]
    \centering
    \includegraphics[width=0.8\textwidth]{plot2b.pdf}
    \caption{Aufgabe 2b: Laufzeiten der einzelnen Schritte doppellogarithmisch aufgetragen.}
    \label{fig:plot2b}
\end{figure}
\FloatBarrier

\subsection*{c) Deutung der Ergebnisse}

\subsubsection*{Gesamtlaufzeit: Abschätzung der Laufzeit für N = 1.000.000}
Offensichtlich besteht ein annähernd linearer Zusammenhang zwischen den logarithmierten Werten von $N$ und $t_{\text{ges}}$. Es gilt also

\begin{equation*}
 \ln(t_{\text{ges}})\, = \, a\,\ln(N)\, +\, b \qquad .
\end{equation*}

\noindent Zur Berechnung der Parameter $a$ und $b$ werden zwei beliebig gewählte Wertepaare eingesetzt und nach $a$ und $b$ umgeformt.

\begin{align*}
  N\,&=\,8192 \qquad t_{\text{ges}} \approx 24 \text{s} \\
  N\,&=\,4096 \qquad t_{\text{ges}} \approx 3 \text{s}\\
  \\
  \Rightarrow a & = \frac{\ln\left(\frac{3}{24}\right)}{\ln\left(\frac{1}{2}\right)} = 3 \\
  b & = \ln(24)-a\, \ln(8192) \approx - 23,85
\end{align*}

Somit ergibt sich für ein Gleichungssystem mit einer 1Mio$\times$1Mio-Matrix eine Laufzeit von

\begin{equation*}
  t_{\text{ges}} = \exp\left(3\ln(10^6)-23,85\right) \text s \approx 4,4 \cdot 10^{7} \text{s} \qquad .
\end{equation*}

\subsubsection*{Optimierungspotential:}
Werden die Laufzeiten nicht logarithmisch aufgetragen, wird noch deutlicher, dass die LU-Zerlegung den größten Anteil der Gesamtlaufzeit ausmacht, wie in Abbildung \ref{fig:plot2c} zu sehen ist. Somit würde es sich am meisten auszahlen die LU-Zelegung bezüglich der Laufzeit zu optimieren.

\FloatBarrier
\begin{figure}[h]
    \centering
    \includegraphics[width=0.8\textwidth]{plot2c.pdf}
    \caption{Aufgabe 2b: Laufzeiten der einzelnen Schritte.}
    \label{fig:plot2c}
\end{figure}
\FloatBarrier

\subsection*{d) Welche Faktoren schränken die Berechnung der Eigenwerte für große Matrizen weiter ein?}
Für sehr große Matrizen ist die LU-Zerlegung anfällig gegenüber Rundungsfehlern und wird somit instabil.

\section*{Aufgabe 3: Profiling zum Vergleich von Algorithmen}
Ziel der Aufgabe ist die Anwendung eines einfachen Profiling zur Untersuchung verschiedener Routinen der
Bibliothek \texttt{Eigen}. Dazu wird, wie auch in Aufgabe 2, ein lineares Gleichungssystem $Mx = b$ mit einer Matrix
$M$ und einer rechten Seite $b$ mit jeweils zufälligen Einträgen erstellt, wofür \texttt{Eigen::MatrixXd::Random} und
\texttt{Eigen::VectorXd::Random} verwendet wird. Die Dimensionalität der Matrix sei N.
\begin{itemize}[leftmargin=*]
\item[a)] Zur Lösung des Gleichungssystems werden nun die folgenden Methoden implementiert und verglichen
\begin{itemize}[leftmargin=*]
  \item[A)] Lösung durch Bestimmung und Anwendung der inversen Matrix $M^{-1}$ mit \\ \texttt{M.inverse()}
  \item[B)] Lösung über eine LU-Zerlegung mit vollständiger Pivotisierung durch \\ \texttt{M.fullPivLu().solve(b)}
  \item[C)] Lösung über eine LU-Zerlegung mit teilweiser Pivotisierung durch \\  \texttt{M.partialPivLu().solve(b)}
\end{itemize}
Zusätzlich wird überprüft ob das System invertierbar ist, wozu die Determinante verwendet wird.
\item[b)] Im Programm wird außerdem überprüft, ob alle Methoden A, B und C die gleichen Ergbnisse liefern. Das ist
jedoch hier auch immer der Fall.
\item[c)] Die Ergebnisse der Laufzeitmessung der verschiedenen Vorgehensweisen sind in Abbildung \ref{fig:plot3} dargestellt.
Verwendet wurde ein linear anwachsendes N mit Maximalwert $\text{N} = 1000$.
\FloatBarrier
\begin{figure}[h]
    \centering
    \includegraphics[width=0.8\textwidth]{a_3.pdf}
    \caption{Aufgabe 3: Vergleich der Laufzeit der verschiedenen  Algorithmen zur Lösung des linearen Gleichungssystems. Verwendet wird eine doppellogarithmische Skalierung.}
    \label{fig:plot3}
\end{figure}
\FloatBarrier
\item[d)] Algorithmenvergleich: Am effizientesten ist in diesem Bereich von N die Berechnung über LU-Zerlegung mit
teilweiser Pivotisierung. Weniger effizient ist die Berechnung über LU-Zerlegung mit voller Pivotisierung
und über die Inverse, wobei der Zeitaufwand für beide Methoden sich wenig unterscheidet. Die Berechnung mit der inversen
Matrix ist jedoch sinnvoll, falls diese noch weiter benötigt wird. Beachtet werden muss auch, dass die teilweise LU-Zerlegung
nur für quadratische invertierbare Matrizen möglich ist. Mit voller Pivotisierung funktioniert die LU-Zerlegung für alle Matrizen und
hat zudem den Vorteil, dass sie genauer ist.
\end{itemize}


\end{document}
