\documentclass[titlepage=firstiscover, captions=tableheading, bibliography=totoc]{scrartcl}
\usepackage[autostyle=true,german=quotes]{csquotes}
\usepackage{scrhack}
\usepackage{enumitem}
\usepackage{caption}
\usepackage[aux]{rerunfilecheck}
\usepackage{subcaption}
\usepackage{fontspec}
\usepackage[dvips]{graphicx}
\usepackage{floatflt,epsfig}

\usepackage{polyglossia}
\setmainlanguage{german}

\usepackage[unicode]{hyperref}
\usepackage{bookmark}
\title{Computational Physics}
\subtitle{Übungsblatt 7}
\author{
Miriam Simm\\
\texorpdfstring{\href{mailto:miriam.simm@tu-dortmund.de}{miriam.simm@tu-dortmund.de}\and}{,}
Katrin Bolsmann\\
\texorpdfstring{\href{mailto:katrin.bolsmann@tu-dortmund.de}{katrin.bolsmann@tu-dortmund.de}}{,}\\
\\
Mario Alex Hollberg\\
\texorpdfstring{\href{mailto:mario-alex.hollberg@tu-dortmund.de}{mario-alex.hollberg@tu-dortmund.de}}{}
}
\date{Abgabe: 15. Mai 2020}
\usepackage{amsmath}
\usepackage{amssymb}
\usepackage{mathtools}
\usepackage[
    math-style=ISO,
    bold-style=ISO,
    sans-style=italic,
    nabla=upright,
    partial=upright,
]{unicode-math}

\setmathfont{Latin Modern Math}

\usepackage[
  locale=DE,
  separate-uncertainty=true,
  per-mode=symbol-or-fraction,
]{siunitx}

\usepackage{multicol}
\setlength{\columnsep}{1pt} %space between columns

\usepackage{booktabs}
\usepackage[x11names, table]{xcolor}
\usepackage{graphicx}
\usepackage{grffile}
\usepackage{xfrac}
\usepackage{xcolor}

\usepackage{float}
\floatplacement{figure}{h}
\floatplacement{table}{h}
\usepackage[
  section,
  below,
]{placeins}

\usepackage{expl3}
\usepackage{xparse}
\ExplSyntaxOn
\NewDocumentCommand \E {} {\symup{e}}
\ExplSyntaxOff

% Literaturverzeichnis
\usepackage[
  backend=biber,
]{biblatex}
% Quellendatenbank
\addbibresource{literatur.bib}

\usepackage[
  version=4,
  math-greek=default,
  text-greek=default,
]{mhchem}


\raggedcolumns


\begin{document}

\maketitle

\section*{Aufgabe 1: Fouriertransformation}

\section*{Aufgabe 2: Eindimensionale Minimierungsverfahren}

Mittels zweier unterschiedlicher Verfahren sollen das Minima der Funktion
\begin{equation*}
  f(x) = x^2 - 2
\end{equation*}

\noindent
gefunden werden. Die dabei benötigten Iterationsschritte werden in Tabelle
\ref{tab:A2} wiedergeben.

\begin{table}
\centering
\begin{tabular}{c c c}
  \toprule
  $\text{Verfahren}$          & $\text{Intervallhalbierung}$ & $\text{Newton}$  \\
  \midrule
  $\text{Iterationsschritte}$ & 61                           & 1                \\
  \bottomrule
\end{tabular}
\caption{Die Genauigkeitsschranke liegt bei $x_{\text{c}} = 10^{-9}$.}
\label{tab:A2}
\end{table}

\noindent
Es wird deutlich, dass das Newton-Verfahren wenseltich weniger Iterationsschritte
benötigt als das Intervallhalbierungs-Verfahren. Somit ist dies (für den Startwert von
$x_0 = 1$) das effizientere Verfahren.


\end{document}
