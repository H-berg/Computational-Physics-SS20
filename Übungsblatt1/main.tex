\documentclass[titlepage=firstiscover, captions=tableheading, bibliography=totoc]{scrartcl}
\usepackage[autostyle=true,german=quotes]{csquotes}
\usepackage{scrhack}
\usepackage{enumitem}
\usepackage{caption}
\usepackage[aux]{rerunfilecheck}
\usepackage{subcaption}
\usepackage{fontspec}
\usepackage[dvips]{graphicx}
\usepackage{floatflt,epsfig}

\usepackage{polyglossia}
\setmainlanguage{german}

\usepackage[unicode]{hyperref}
\usepackage{bookmark}
\title{Computational Physics}
\subtitle{Übungsblatt 4}
\author{
Miriam Simm\\
\texorpdfstring{\href{mailto:miriam.simm@tu-dortmund.de}{miriam.simm@tu-dortmund.de}\and}{,}
Katrin Bolsmann\\
\texorpdfstring{\href{mailto:katrin.bolsmann@tu-dortmund.de}{katrin.bolsmann@tu-dortmund.de}}{,}\\
\\
Mario Alex Hollberg\\
\texorpdfstring{\href{mailto:mario-alex.hollberg@tu-dortmund.de}{mario-alex.hollberg@tu-dortmund.de}}{}
}
\date{Abgabe: 22. Mai 2020}
\usepackage{amsmath}
\usepackage{amssymb}
\usepackage{mathtools}
\usepackage[
    math-style=ISO,
    bold-style=ISO,
    sans-style=italic,
    nabla=upright,
    partial=upright,
]{unicode-math}

\setmathfont{Latin Modern Math}

\usepackage[
  locale=DE,
  separate-uncertainty=true,
  per-mode=symbol-or-fraction,
]{siunitx}

\usepackage{multicol}
\setlength{\columnsep}{1pt} %space between columns

\usepackage{booktabs}
\usepackage[x11names, table]{xcolor}
\usepackage{graphicx}
\usepackage{grffile}
\usepackage{xfrac}
\usepackage{xcolor}

\usepackage{float}
\floatplacement{figure}{h}
\floatplacement{table}{h}
\usepackage[
  section,
  below,
]{placeins}

\usepackage{expl3}
\usepackage{xparse}
\ExplSyntaxOn
\NewDocumentCommand \E {} {\symup{e}}
\ExplSyntaxOff

% Literaturverzeichnis
\usepackage[
  backend=biber,
]{biblatex}
% Quellendatenbank
\addbibresource{literatur.bib}

\usepackage[
  version=4,
  math-greek=default,
  text-greek=default,
]{mhchem}

\def\Xint#1{\mathchoice
   {\XXint\displaystyle\textstyle{#1}}%
   {\XXint\textstyle\scriptstyle{#1}}%
   {\XXint\scriptstyle\scriptscriptstyle{#1}}%
   {\XXint\scriptscriptstyle\scriptscriptstyle{#1}}%
   \!\int}
\def\XXint#1#2#3{{\setbox0=\hbox{$#1{#2#3}{\int}$}
     \vcenter{\hbox{$#2#3$}}\kern-.5\wd0}}
\def\ddashint{\Xint=}
\def\dashint{\Xint-}


\raggedcolumns


\begin{document}

\maketitle

\section*{Aufgabe1}

\subsection*{a)}
Die Basen deuten auf ein hexagonales Kristallsystem hin.

\subsection*{b)}

Geben sind drei Basis-Vektoren $\vec{a_1}$, $\vec{a_2}$ und $\vec{a_3}$, die
zusammen die Matrix $A = \biggl(\vec{a_1}\;\vec{a_2}\;\vec{a_3}\biggr)$ ergeben:

\begin{equation*} %added with Mathpix Snipping-Tool
  \vec{a}_{1}=\left(\begin{array}{c}
\frac{1}{2} \\
\frac{\sqrt{3}}{2} \\
0
\end{array}\right) ; \quad \vec{a}_{2}=\left(\begin{array}{c}
-\frac{1}{2} \\
\frac{\sqrt{3}}{2} \\
0
\end{array}\right) ; \quad \vec{a}_{3}=\left(\begin{array}{l}
0 \\
0 \\
1
\end{array}\right)
\implies
A =
\left(\begin{array}{rrr}
  \frac{1}{2}         & -\frac{1}{2}         & 0 \\
  \frac{\sqrt{3}}{2}  & \frac{\sqrt{3}}{2}  & 0 \\
  0                   & 0                   & 1 \\
\end{array}\right)
\end{equation*}

\noindent
Eine Fehlstelle befindet sich bei $\vec{x}=(2,0,2)^{T}$ %added with Mathpix Snipping-Tool
und es sollen nun die Koordinaten $\vec{x'}$ bezüglich der Basen $\vec{a_1}$, $\vec{a_2}$ und $\vec{a_3}$
bestimmten werden. Dazu wird folgendes GLS mit Bibliothek $\textit{Eigen}$ gelöst:

\begin{equation*}
  A\vec{x'} = \vec{x}
\end{equation*}

\noindent
wobei die Matrix $A$ mit $\textit{Eigen:PartialPivLu}$ zerlegt wird: $A=PLU$ (LU-Zerlegung).
Bei der Zerlegung ergeben sich eine Permutations-Matrix $P$, eine Lower-triangular-Matrix
$L$ und eine Upper-triangular-Matrix $U$:

\begin{equation*}
  P_A =
  \left(\begin{array}{rrr}
    0 & 1 & 0 \\
    1 & 0 & 0 \\
    0 & 0 & 1 \\
  \end{array}\right),
  \quad
  L_A =
  \left(\begin{array}{rrr}
    1       &  0 & 0 \\
    0.57735 &  1 & 0 \\
    0       &  0 & 1 \\
  \end{array}\right),
  \quad
  U_A =
  \left(\begin{array}{rrr}
  0.866025 & 0.866025 & 0 \\
         0 &       -1 & 0 \\
         0 &        0 & 1 \\
  \end{array}\right)
\end{equation*}

\noindent
Somit erhalten wir von der $\textit{solve}$-Funktion
folgende Koordinaten:

\begin{equation*}
  \vec{x'} = (2,-2,2)^T
\end{equation*}

\subsection*{c)}
Erneuert sollen zunächst die Koordinaten $\vec{y'}$ einer Fehlstelle bei
$\vec{y}=(1,2 \sqrt{3}, 3)^{T}$ %added with Mathpix Snipping-Tool
bestimmt werden:

\begin{equation*}
  \vec{y'} = (3,1,3)^T
\end{equation*}

\noindent
Die LU-Zerlegung von $A=PLU$ kann als Zwischenergebnis benutzt werden, sodass
nur mittels der $\textit{solve}$-Funktion das LGS gelöst werden kann.

\subsection*{d)}

Erneuert wird eine LU-Zerlegung gemacht, allerdings mit der Matrix
$B = \biggl(\vec{a_3}\;\vec{a_2}\;\vec{a_1}\biggr) =
\left(\begin{array}{rrr}
  0  & \frac{1}{2}         & -\frac{1}{2}       \\
  0  & \frac{\sqrt{3}}{2}  & \frac{\sqrt{3}}{2} \\
  1  & 0                   & 0                  \\
\end{array}\right)$. Es ergeben sich die Matrizen:

\begin{equation*}
  P_B =
  \left(\begin{array}{rrr}
    0 & 0 & 1 \\
    0 & 1 & 0 \\
    1 & 0 & 0 \\
  \end{array}\right),
  \quad
  L_B =
  \left(\begin{array}{rrr}
    1 &  0          & 0 \\
    0 &  1          & 0 \\
    0 & -0.57735    & 1 \\
  \end{array}\right),
  \quad
  U_B =
  \left(\begin{array}{rrr}
    1 & 0           & 0        \\
    0 & 0.866025    & 0.866025 \\
    0 & 0           & 1        \\
  \end{array}\right)
\end{equation*}

\noindent
Im Vergleich zu den $P$, $L$ und $U$ Martizen aus b) fällt auf besonders die $P$-Matrix auf.
Offensichtlich werden in der $\textit{Eigen:PartialPivLu}$-Modul zunächst
die (betragsmäßig) größte Zahl getauscht. Vergleiche erste Spalte der beiden $P$-Matrizen:\\
$P_A$: Die erste Zeile ($\frac{1}{2}$-Eintrag) wird mit der dritten Zeile
($\frac{\sqrt{3}}{2}$-Eintrag) getauscht. Die letzte Spalte braucht hingegen nicht mehr getausch werden.\\
$P_B$: Die erste Zeile (Null-Eintrag) wird mit der dritten Zeile ($1$-Eintrag) getauscht.
Die mittlere Spalte bleibt an ihrem Ort.\\
Das Vorgehen soll absichern, dass durch keine Null geteilt wird. Effektiv werden
$A$ und $B$ aber gleich oft getauscht. Dies führt zu unterschiedlichen $L$ und
$U$ Matrizen.

\section*{Aufgabe2: Ausgleichsrechnung}

Gegeben seien die folgenden (x, y)-Datenpunkte:
\FloatBarrier
\begin{table}[h]
    \centering
    \begin{tabular}{S S S S S S S S S S S}
      \toprule
        \textbf{x} | & 0 & 2,5 & -6,3 & 4 & -3,2 & 5,3 & 10,1 & 9,5 & -5,4 & 12,7 \\
      \midrule
        \textbf{y} | & 4 & 4,3 & -3,9 & 6,5 & 0,7 & 8,6 & 13 & 9,9 & -3,6 & 15,1 \\
        \bottomrule
    \end{tabular}
\end{table}
\noindent
Für diese soll eine Ausgleichsgerade
\begin{align*}
  y(x)&=mx\,+\,n \\
\intertext{berechnet werden, welche den quadratischen Fehler}\\
  R& = \sum_i(m\, x_i \, +\, n \, -\, y_i)^2\\
\end{align*}
minimiert.




\subsection*{a) Formulierung des überbestimmten Gleichungssystems}

R soll minimiert werden
\begin{align*}
  \vec{0} \stackrel{!}{=}\vec{\nabla}_{m,n}\cdot R &=\left(\begin{array}{c} \sum_i 2(mx_i+n-y_i)x_i \\ \sum_i 2(mx_i+n-y_i) \end{array}\right)\\ \\
  \Leftrightarrow \sum_i(m x_i + n) & = \sum_i y_i \\
\end{align*}
Somit lautet das überbestimmte Gleichungssystem
\begin{equation*}
  \Rightarrow \underbrace{\left(\vec{x} \quad  \vec{1} \right)}_{A}\underbrace{\left(\begin{array}{c} m \\ n \end{array}\right)}_{\vec{n}} = \vec{y} \qquad .
\end{equation*}
\noindent
Wobei $\vec{1}$ ein Vektor ist, welcher nur Einsen beinhaltet und die gleiche Dimension wie $\vec{x}$ hat. Somit handelt es sich bei A um eine $10\times 2$ - Matrix.

\subsection*{b) Überführung in ein quadratisches Problem}

Mit der Matrix $P = A^{T}A$ kann das Systen folgendermaßen in ein quadratisches Gleichungssystem überführt werden:
\begin{align*}
  A\,\vec{n}&=\vec{y} \qquad |\cdot A^{T} \\
\Leftrightarrow  P\,\vec{n}&=A^{T}\,\vec{b}
\end{align*}

\subsection*{c) Lösen des Gleichungssystems}
Das symmetrische Problem kann nun mittels einer LU-Zerlegung gelöst werden (siehe Aufgabe2.cpp).
Für die Steigung $m$ und den y-Achsenabschnitt $n$ ergeben sich die Werte
\begin{align*}
  m&=0.959951\\
  n&=2.65694
\end{align*}

\subsection*{d) Graphische Darstellung der Ausgleichsgeraden}

\FloatBarrier
\begin{figure}[h]
    \centering
    \includegraphics[width=0.8\textwidth]{plot2d.pdf}
    \caption{Aufgabe 2d: Ausgleichsgerade mit der Steigung $m=0.959951$ und dem y-Achsenabschnitt $n=2.65694$}
    \label{fig:plot2b}
\end{figure}
\FloatBarrier

\end{document}
