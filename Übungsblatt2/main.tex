\documentclass[titlepage=firstiscover, captions=tableheading, bibliography=totoc]{scrartcl}
\usepackage[autostyle=true,german=quotes]{csquotes}
\usepackage{scrhack}
\usepackage{enumitem}
\usepackage{caption}
\usepackage[aux]{rerunfilecheck}
\usepackage{subcaption}
\usepackage{fontspec}
\usepackage[dvips]{graphicx}
\usepackage{floatflt,epsfig}

\usepackage{polyglossia}
\setmainlanguage{german}

\usepackage[unicode]{hyperref}
\usepackage{bookmark}
\title{Computational Physics}
\subtitle{Übungsblatt 4}
\author{
Miriam Simm\\
\texorpdfstring{\href{mailto:miriam.simm@tu-dortmund.de}{miriam.simm@tu-dortmund.de}\and}{,}
Katrin Bolsmann\\
\texorpdfstring{\href{mailto:katrin.bolsmann@tu-dortmund.de}{katrin.bolsmann@tu-dortmund.de}}{,}\\
\\
Mario Alex Hollberg\\
\texorpdfstring{\href{mailto:mario-alex.hollberg@tu-dortmund.de}{mario-alex.hollberg@tu-dortmund.de}}{}
}
\date{Abgabe: 22. Mai 2020}
\usepackage{amsmath}
\usepackage{amssymb}
\usepackage{mathtools}
\usepackage[
    math-style=ISO,
    bold-style=ISO,
    sans-style=italic,
    nabla=upright,
    partial=upright,
]{unicode-math}

\setmathfont{Latin Modern Math}

\usepackage[
  locale=DE,
  separate-uncertainty=true,
  per-mode=symbol-or-fraction,
]{siunitx}

\usepackage{multicol}
\setlength{\columnsep}{1pt} %space between columns

\usepackage{booktabs}
\usepackage[x11names, table]{xcolor}
\usepackage{graphicx}
\usepackage{grffile}
\usepackage{xfrac}
\usepackage{xcolor}

\usepackage{float}
\floatplacement{figure}{h}
\floatplacement{table}{h}
\usepackage[
  section,
  below,
]{placeins}

\usepackage{expl3}
\usepackage{xparse}
\ExplSyntaxOn
\NewDocumentCommand \E {} {\symup{e}}
\ExplSyntaxOff

% Literaturverzeichnis
\usepackage[
  backend=biber,
]{biblatex}
% Quellendatenbank
\addbibresource{literatur.bib}

\usepackage[
  version=4,
  math-greek=default,
  text-greek=default,
]{mhchem}

\def\Xint#1{\mathchoice
   {\XXint\displaystyle\textstyle{#1}}%
   {\XXint\textstyle\scriptstyle{#1}}%
   {\XXint\scriptstyle\scriptscriptstyle{#1}}%
   {\XXint\scriptscriptstyle\scriptscriptstyle{#1}}%
   \!\int}
\def\XXint#1#2#3{{\setbox0=\hbox{$#1{#2#3}{\int}$}
     \vcenter{\hbox{$#2#3$}}\kern-.5\wd0}}
\def\ddashint{\Xint=}
\def\dashint{\Xint-}


\raggedcolumns


\begin{document}

\maketitle

\section*{Aufgabe1: Singulärwertzerlegung}

\section*{Aufgabe2: Profiling zur Untersuchung eines Algorithmus}
In dieser Aufgabe soll ein zufälliges Lineares Gleichungssystem, aus einer quadratischen Matrix $M$ der Dimension $N$ und N-dimensionalen Vektoren $x$ und $b$

\begin{equation*}
  M\, x\, =\, b
\end{equation*}

\noindent
mit einer LU-Zerlegung gelöst werden. Dabei soll die Laufzeit der einzelnen Arbeitsschritte

\begin{itemize}
  \item[1)] Erstellen einer zufälligen N-dimensionalen Matrix $M$
  \item[2)] Durchführung der LU-Zerlegung
  \item[3)] Lösen des Gleichungssystems mit LU-Zerlegung
\end{itemize}

\noindent
für verschiedene Dimensionen $N$ verglichen werden.

\subsection*{b) Plotten der Laufzeiten}
In diesem Aufgabenteil wurde N logarithmisch vergrößert und die Laufzeiten gemessen. Diese sind in Plot \ref{fig:plot2b} doppellogarithmisch aufgetragen. Zusätzlich wurde auch die Gesamtlaufzeit des Algorithmus berechnet.

\FloatBarrier
\begin{figure}[h]
    \centering
    \includegraphics[width=0.8\textwidth]{plot2b.pdf}
    \caption{Aufgabe 2b: Laufzeiten der einzelnen Schritte doppellogarithmisch aufgetragen.}
    \label{fig:plot2b}
\end{figure}
\FloatBarrier

\subsection*{c) Deutung der Ergebnisse}

\subsubsection*{Gesamtlaufzeit: Abschätzung der Laufzeit für N = 1.000.000}
Offensichtlich besteht ein annähernd linearer Zusammenhang zwischen den logarithmierten Werten von $N$ und $t_{\text{ges}}$. Es gilt also

\begin{equation*}
 \ln(t_{\text{ges}})\, = \, a\,\ln(N)\, +\, b \qquad .
\end{equation*}

\noindent Zur Berechnung der Parameter $a$ und $b$ werden zwei beliebig gewählte Wertepaare eingesetzt und nach $a$ und $b$ umgeformt.

\begin{align*}
  N\,&=\,8192 \qquad t_{\text{ges}} \approx 24 \text{s} \\
  N\,&=\,4096 \qquad t_{\text{ges}} \approx 3 \text{s}\\
  \\
  \Rightarrow a & = \frac{\ln\left(\frac{3}{24}\right)}{\ln\left(\frac{1}{2}\right)} = 3 \\
  b & = \ln(24)-a\, \ln(8192) \approx - 23,85
\end{align*}

Somit ergibt sich für ein Gleichungssystem mit einer 1Mio$\times$1Mio-Matrix eine Laufzeit von

\begin{equation*}
  t_{\text{ges}} = \exp\left(3\ln(10^6)-23,85\right) \text s \approx 4,4 \cdot 10^{7} \text{s} \qquad .
\end{equation*}

\subsubsection*{Optimierungspotential:}
Werden die Laufzeiten nicht logarithmisch aufgetragen, wird noch deutlicher, dass die LU-Zerlegung den größten Anteil der Gesamtlaufzeit ausmacht, wie in Abbildung \ref{fig:plot2c} zu sehen ist. Somit würde es sich am meisten auszahlen die LU-Zelegung bezüglich der Laufzeit zu optimieren.

\FloatBarrier
\begin{figure}[h]
    \centering
    \includegraphics[width=0.8\textwidth]{plot2c.pdf}
    \caption{Aufgabe 2b: Laufzeiten der einzelnen Schritte.}
    \label{fig:plot2c}
\end{figure}
\FloatBarrier

\subsection*{d) Welche Faktoren schränken die Berechnung der Eigenwerte für große Matrizen weiter ein?}
Für sehr große Matrizen ist die LU-Zerlegung anfällig gegenüber Rundungsfehlern und wird somit instabil.

\section*{Aufgabe3: Profiling zur Untersuchung von Algorithmen}

\end{document}
