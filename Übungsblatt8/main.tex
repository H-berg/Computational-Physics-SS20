\documentclass[titlepage=firstiscover, captions=tableheading, bibliography=totoc]{scrartcl}
\usepackage[autostyle=true,german=quotes]{csquotes}
\usepackage{scrhack}
\usepackage{enumitem}
\usepackage{caption}
\usepackage[aux]{rerunfilecheck}
\usepackage{subcaption}
\usepackage{fontspec}
\usepackage[dvips]{graphicx}
\usepackage{floatflt,epsfig}

\usepackage{polyglossia}
\setmainlanguage{german}

\usepackage[unicode]{hyperref}
\usepackage{bookmark}
\title{Computational Physics}
\subtitle{Übungsblatt 4}
\author{
Miriam Simm\\
\texorpdfstring{\href{mailto:miriam.simm@tu-dortmund.de}{miriam.simm@tu-dortmund.de}\and}{,}
Katrin Bolsmann\\
\texorpdfstring{\href{mailto:katrin.bolsmann@tu-dortmund.de}{katrin.bolsmann@tu-dortmund.de}}{,}\\
\\
Mario Alex Hollberg\\
\texorpdfstring{\href{mailto:mario-alex.hollberg@tu-dortmund.de}{mario-alex.hollberg@tu-dortmund.de}}{}
}
\date{Abgabe: 22. Mai 2020}
\usepackage{amsmath}
\usepackage{amssymb}
\usepackage{mathtools}
\usepackage[
    math-style=ISO,
    bold-style=ISO,
    sans-style=italic,
    nabla=upright,
    partial=upright,
]{unicode-math}

\setmathfont{Latin Modern Math}

\usepackage[
  locale=DE,
  separate-uncertainty=true,
  per-mode=symbol-or-fraction,
]{siunitx}

\usepackage{multicol}
\setlength{\columnsep}{1pt} %space between columns

\usepackage{booktabs}
\usepackage[x11names, table]{xcolor}
\usepackage{graphicx}
\usepackage{grffile}
\usepackage{xfrac}
\usepackage{xcolor}

\usepackage{float}
\floatplacement{figure}{h}
\floatplacement{table}{h}
\usepackage[
  section,
  below,
]{placeins}

\usepackage{expl3}
\usepackage{xparse}
\ExplSyntaxOn
\NewDocumentCommand \E {} {\symup{e}}
\ExplSyntaxOff

% Literaturverzeichnis
\usepackage[
  backend=biber,
]{biblatex}
% Quellendatenbank
\addbibresource{literatur.bib}

\usepackage[
  version=4,
  math-greek=default,
  text-greek=default,
]{mhchem}

\def\Xint#1{\mathchoice
   {\XXint\displaystyle\textstyle{#1}}%
   {\XXint\textstyle\scriptstyle{#1}}%
   {\XXint\scriptstyle\scriptscriptstyle{#1}}%
   {\XXint\scriptscriptstyle\scriptscriptstyle{#1}}%
   \!\int}
\def\XXint#1#2#3{{\setbox0=\hbox{$#1{#2#3}{\int}$}
     \vcenter{\hbox{$#2#3$}}\kern-.5\wd0}}
\def\ddashint{\Xint=}
\def\dashint{\Xint-}


\raggedcolumns


\begin{document}

\maketitle

\section*{Molekulardynamik-Simulation eines $2d$-Lennard-Jones-Fluids}
\subsection*{Lennard-Jones-Wechselwirkung}
\label{section: Einleitung}
Ziel dieser Aufgabe ist die Simulation der paarweisen Lennard-Jones-Wechselwirkung
\begin{equation*}
    V \left(r\right) = 4 \left[\left(\frac{1}{r}\right)^{12} - \left(\frac{1}{r}\right)^{6}\right]
\end{equation*}
für $N$ identische Teilchen mit Masse $m = 1$. Bei dieser Form des Potentials werden Längen in Einheiten von $\sigma$
und Energien und $k_B T$ in Einheiten von $\varepsilon$ gemessen. 
Es wird ein zweidimensionales System der festen Größe $A = L \times L$ mit
periodischen Randbedingungen mit  $N = 16$ Teilchen betrachtet. Die Simulation erfolgt für ein mikrokanonisches System,
es muss daher die Gesamtenergie erhalten sein.

\subsection*{Initialisierung}
Vor Beginn der Simulation erfolgt eine Initialisierung. Dazu werden die Teilchen mit gleichmäßigem Abstand auf Plätzen
\begin{equation*}
    \vec{r} \left(0\right) = \frac{L}{8} \left(1 + 2 n, 1 + 2 m\right) \qquad m, n \in \left\{0, \dots, 3\right\}
\end{equation*}
in der Box $\left[0, L\right] \times \left[0, L\right]$ verteilt.
Für die Anfangsgeschwindigkeiten gilt die Bedingung, dass die Schwerpunktsgeschwindigkeit zu Anfang gleich 0 ist,
also
\begin{equation*}
    \sum_{i = 1}^{N} \vec{v}_i \left(0\right) = \vec{0} \, ,
\end{equation*}
da nur die Relativgeschwindigkeiten der einzelnen Teilchen zueinander physikalisch interessant sind.
Zudem wird eine Umskalierung $\vec{v} \rightarrow \alpha \vec{v}$ der Geschwindigkeiten implementiert, damit bei einer
beliebigen Anfangstemperatur $T \left(t = 0\right)$ begonnen werden kann. 
Dazu werden die Anfangsgeschwindigkeiten zuerst zufällig gewählt, wozu
die \texttt{Eigen}-Funktion \texttt{MatrixXd::Random} verwendet wird, und dann die Schwerpunktsgeschwindigkeit abgezogen
\begin{equation*}
    \vec{v}_i \rightarrow \vec{v}_i - \frac{1}{N} \sum_{j = 1}^{N} \vec{v}_j \, .
\end{equation*}
Zur Bestimmung des Skalierungsfaktors $\alpha$ wird die Temperatur $T$ betrachtet 
\begin{align*}
    T &= \frac{2}{k_B T} \left\langle \frac{\vec{p}^2}{2 m}\right\rangle \\
      &= \frac{2}{k_B T} \sum_{i = 1}^{N} \frac{\vec{p}_i^2}{2 m} \\
      &= \frac{2}{k_B T} \sum_{i = 1}^{N} \frac{\alpha^2 \vec{\tilde{p}}_i^2}{2 m} \\
      &= \alpha^2 \tilde{T} \, ,
\end{align*}
wobei $\tilde{T}$ die gewünschte Temperatur ist.
Der Skalierungsfaktor ergibt sich dann durch
\begin{equation*}
    \alpha = \sqrt{\frac{T}{\tilde{T}}}
\end{equation*}
mit 
\begin{equation*}
    \tilde{T} = \frac{1}{k_b N_f} \sum_{i = 1}^{N} m v^2 \,
\end{equation*}
mit der Anzahl der Freiheitsgrade $N_f = 2N - 2$ in einem zweidimensionalen System.

\subsection*{Äquilibrierung}
In der Äquilibrierungsphase geht das System in den thermodynamischen Gleichgewichtszustand über.
Begonnen wird mit $T(0) = 0$. Die Lösung der Newton'schen Bewegungsgleichungen erfolgt mit dem Geschwindigkeits-Verlet-Algorithmus
\begin{align*}
    \vec{r}_{n+1} &= \vec{r}_n + \vec{v}_n + \frac{1}{2} \vec{a}_n h^2 \\
    \vec{v}_{n+1} &= \vec{v}_n + \frac{h}{2} \left(\vec{a}_{n+1} - \vec{a}_{n}\right)
\end{align*}
mit dem Zeitschritt $h$ für den hier $h \leq 0.01$ verwendet wird. 

Da $m = 1$ gewählt wird, ergibt sich die Beschleunigung $\vec{a}$ direkt aus der berechneten Kraft $\vec{F}$.
Als Cutoff wird $r_c = L/2$ gewählt. Im hier vorliegenden Potential berechnet sich die Kraft auf ein Teilchen $i$ gemäß
\begin{equation*}
    \vec{F}_{i} = - \sum_{i \neq j} \sum_{\vec{n} \in \mathbb{Z}^3} \frac{\vec{r}_{ij} + \vec{L} \vec{n}}{\left|\vec{r}_{ij} + \vec{L} \vec{n}\right|} U'\left(\left|\vec{r}_{ij} + \vec{L} \vec{n}\right|\right)
\end{equation*}
mit
\begin{equation*}
    \vec{L} \left(\vec{n}\right) = \left(n_x L_x, n_y L_y, n_z L_z\right)^{\text{T}}
\end{equation*}
und 
\begin{equation*}
    U' \left(\vec{r}\right) = 48 \frac{\vec{r}}{\left|\vec{r}\right|} \left[\left(\frac{1}{r}\right)^{13} - \frac{1}{2} \left(\frac{1}{r}\right)^{7}\right] \, .
\end{equation*}
In der Funktion \texttt{Funktion einfügen :)} wird verwendet, dass der Betrag der Kraft symmetrisch ist, sodass
die Kraft für Teilchen $i$ und $j$ zu $i$ addiert und von $j$ subtrahiert wird. 

Mit den Randbedingungen werden Teilchen, die sich auf einer Seite aus dem Simulationsgebiet $A$ herausbewegen, wieder auf der anderen Seite 
hineingeführt.
Für der Implementierung in \texttt{Funktion einfügen :)} wird die \texttt{floor}-Funktion verwendet, mit der sich 
für die $x$- und $y$-Komponenten
\begin{align*}
    x' &= x - L \cdot \text{floor}\left(\frac{x}{L}\right) \\
    y' &= y - L \cdot \text{floor}\left(\frac{y}{L}\right) 
\end{align*}
ergibt.

Während der Äquilibrierung werden nun verschiedene Observablen als Funktion der Zeit berechnet.
\subsubsection*{Schwerpunktsgeschwindigkeit $\vec{v}$}
Die Schwerpunktsgeschwindigkeit berechnet sich in jedem Zeitschritt gemäß
\begin{equation*}
    \vec{v} = \frac{1}{N} \sum_{i = 1}^{N} \vec{v}_i \, 
\end{equation*}
sodass sich der folgende Verlauf ergibt.
\FloatBarrier
\begin{figure}[H]
    \centering
    %\includegraphics[width=0.8\textwidth]{.pdf}
    \caption{Verlauf der Schwerpunktsgeschwindigkeit in Abhängigkeit von der Zeit $t$.}
    \label{fig:v}
\end{figure}
\FloatBarrier
\noindent
\subsubsection*{Temperatur $T$}
Die Temperatur berechnet sich nach
\begin{equation*}
    T = \frac{2 E_{kin}}{N_f} \, .
\end{equation*}
Ihr Verlauf ist in Abbildung \ref{fig:T} dargestellt.
\FloatBarrier
\begin{figure}[H]
    \centering
    %\includegraphics[width=0.8\textwidth]{.pdf}
    \caption{Verlauf der Temperatur $T$ in Abhängigkeit von der Zeit $t$.}
    \label{fig:T}
\end{figure}
\FloatBarrier
\noindent
\subsubsection*{Potentielle Energie $E_{pot}$ und kinetische Energie $E_{kin}$}
Der Verlauf der potentiellen Energie 
\begin{equation*}
    E_{pot}(t) = \sum_{i < j-1}^{N} V\left(\left|\vec{r}_i - \vec{r}_j\right|\right)
\end{equation*}
ist zusammen mit dem Verlauf der kinetischen Energie
\begin{equation*}
   E_{kin}(t) = \sum_{i = 1}^{N} \frac{\vec{v}_i^2}{2} \, ,
\end{equation*}
in Abbildung \ref{fig:E} dargestellt.
\FloatBarrier
\begin{figure}[H]
    \centering
    %\includegraphics[width=0.8\textwidth]{.pdf}
    \caption{Verlauf der potentiellen und kinetischen Energie sowie der Gesamtenergie in Abhängigkeit von der Zeit $t$.}
    \label{fig:V}
\end{figure}
\FloatBarrier
\noindent
Am Verlauf der Kurve der Gesamtenergie, die hier zusätzlich eingetragen ist und die konstant 0 ist, ist gut erkennbar, dass die Energie im System erhalten ist.
Die in Abschnitt \ref{section: Einleitung} aufgestellte Forderung an die Simulation ist somit erfüllt.


\subsection*{Messung}
Nach der Äquilibrierungsphase wird eine Mittelung über xy (ergänzen :) Zeitschritte durchgeführt und die Temperatur $T$
sowie die Paarkorrelationsfunktion $g(r)$ gemessen.
Die Messung der Paarkorrelationsfunktion erfolgt über ein Histogramm der Paarabstände für $0 \leq g(\vec{r}) \leq \frac{L}{2}$.
Dieser Bereich wird in $N_H$ Bins der Länge
\begin{equation*}
    \Delta r = \frac{L}{2 N_H}
\end{equation*} 
eingeteilt. Zu jedem Zeitpunkt $t$ wird dann 
für jeden Bin $L$ die Anzahl $P_L(t)$ der Teilchen, die sich in diesem Bin befinden, ermittelt.
Die Paarkorrelationsfunktion berechnet sich dann gemäß
\begin{equation*}
    g \left(\vec{r}_L\right) = \frac{\left\langle P_L \right\rangle}{N \rho \Delta V_L}
\end{equation*}
mit $\rho =  \frac{N}{V}$ .
In unserem Programm ist sie in der Funktion \texttt{Funktionsnamen ergänzen :)} implementiert. 

Die Messung wird nun für $N = 16$ und $L = 8 \sigma$ für die drei verschiedene Anfangstemperaturen 
\begin{align*}
    T(0) &= 1 \\
    T(0) &= 0.01 \\
    T(0) &= 100 
\end{align*}
durchgeführt.
Der Verlauf der Paarkorrelationsfunktion für die verschiedenen Anfangstemperaturen ist in Abbildungen \ref{fig:p1}, \ref{fig:p2} und \ref{fig:p3}
dargestellt.
\FloatBarrier
\begin{figure}[H]
    \centering
    %\includegraphics[width=0.5\textwidth]{.pdf}
    \caption{Verlauf der Paarkorrelationsfunktion $g(\vec{r})$ für die Anfangstemperatur $T(0) = 1$.}
    \label{fig:p1}
\end{figure}
\FloatBarrier
\noindent
\FloatBarrier
\begin{figure}[H]
    \centering
    %\includegraphics[width=0.5\textwidth]{.pdf}
    \caption{Verlauf der Paarkorrelationsfunktion $g(\vec{r})$ für die Anfangstemperatur $T(0) = 0.01$.}
    \label{fig:p2}
\end{figure}
\FloatBarrier
\noindent
\FloatBarrier
\begin{figure}[H]
    \centering
    %\includegraphics[width=0.5\textwidth]{.pdf}
    \caption{Verlauf der Paarkorrelationsfunktion $g(\vec{r})$ für die Anfangstemperatur $T(0) = 100$.}
    \label{fig:p3}
\end{figure}
\FloatBarrier
\noindent
Der Verlauf der Temperatur für die verschiedenen Anfangstemperaturen ist in Abbildungen \ref{fig:t1}, \ref{fig:t2} und \ref{fig:t3}
\FloatBarrier
\begin{figure}[H]
    \centering
    %\includegraphics[width=0.5\textwidth]{.pdf}
    \caption{Verlauf der Temperatur $T$ für die Anfangstemperatur $T(0) = 1$.}
    \label{fig:p1}
\end{figure}
\FloatBarrier
\noindent
\FloatBarrier
\begin{figure}[H]
    \centering
    %\includegraphics[width=0.5\textwidth]{.pdf}
    \caption{Verlauf der Temperatur $T$ für die Anfangstemperatur $T(0) = 0.01$.}
    \label{fig:p2}
\end{figure}
\FloatBarrier
\noindent
\FloatBarrier
\begin{figure}[H]
    \centering
    %\includegraphics[width=0.5\textwidth]{.pdf}
    \caption{Verlauf der Temperatur $T$ für die Anfangstemperatur $T(0) = 100$.}
    \label{fig:p3}
\end{figure}
\FloatBarrier
\noindent

\end{document}
