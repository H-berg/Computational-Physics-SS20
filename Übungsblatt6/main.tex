\documentclass[titlepage=firstiscover, captions=tableheading, bibliography=totoc]{scrartcl}
\usepackage[autostyle=true,german=quotes]{csquotes}
\usepackage{scrhack}
\usepackage{enumitem}
\usepackage{caption}
\usepackage[aux]{rerunfilecheck}
\usepackage{subcaption}
\usepackage{fontspec}
\usepackage[dvips]{graphicx}
\usepackage{floatflt,epsfig}

\usepackage{polyglossia}
\setmainlanguage{german}

\usepackage[unicode]{hyperref}
\usepackage{bookmark}
\title{Computational Physics}
\subtitle{Übungsblatt 4}
\author{
Miriam Simm\\
\texorpdfstring{\href{mailto:miriam.simm@tu-dortmund.de}{miriam.simm@tu-dortmund.de}\and}{,}
Katrin Bolsmann\\
\texorpdfstring{\href{mailto:katrin.bolsmann@tu-dortmund.de}{katrin.bolsmann@tu-dortmund.de}}{,}\\
\\
Mario Alex Hollberg\\
\texorpdfstring{\href{mailto:mario-alex.hollberg@tu-dortmund.de}{mario-alex.hollberg@tu-dortmund.de}}{}
}
\date{Abgabe: 22. Mai 2020}
\usepackage{amsmath}
\usepackage{amssymb}
\usepackage{mathtools}
\usepackage[
    math-style=ISO,
    bold-style=ISO,
    sans-style=italic,
    nabla=upright,
    partial=upright,
]{unicode-math}

\setmathfont{Latin Modern Math}

\usepackage[
  locale=DE,
  separate-uncertainty=true,
  per-mode=symbol-or-fraction,
]{siunitx}

\usepackage{multicol}
\setlength{\columnsep}{1pt} %space between columns

\usepackage{booktabs}
\usepackage[x11names, table]{xcolor}
\usepackage{graphicx}
\usepackage{grffile}
\usepackage{xfrac}
\usepackage{xcolor}

\usepackage{float}
\floatplacement{figure}{h}
\floatplacement{table}{h}
\usepackage[
  section,
  below,
]{placeins}

\usepackage{expl3}
\usepackage{xparse}
\ExplSyntaxOn
\NewDocumentCommand \E {} {\symup{e}}
\ExplSyntaxOff

% Literaturverzeichnis
\usepackage[
  backend=biber,
]{biblatex}
% Quellendatenbank
\addbibresource{literatur.bib}

\usepackage[
  version=4,
  math-greek=default,
  text-greek=default,
]{mhchem}

\def\Xint#1{\mathchoice
   {\XXint\displaystyle\textstyle{#1}}%
   {\XXint\textstyle\scriptstyle{#1}}%
   {\XXint\scriptstyle\scriptscriptstyle{#1}}%
   {\XXint\scriptscriptstyle\scriptscriptstyle{#1}}%
   \!\int}
\def\XXint#1#2#3{{\setbox0=\hbox{$#1{#2#3}{\int}$}
     \vcenter{\hbox{$#2#3$}}\kern-.5\wd0}}
\def\ddashint{\Xint=}
\def\dashint{\Xint-}


\raggedcolumns


\begin{document}

\maketitle

\section*{Aufgabe 1: Mehrdimensionale Minimierung}
Ziel der Aufgabe ist die Implementierung und Überprüfung des Gradienten-Verfahrens und des Konjuguerte-Gradienten-Verfahrens zur mehrdimensionalen Mininmierung.
\begin{itemize}[leftmargin=*]
\item[a)] Für die Implementierung des Gradienten-Verfahrens und des Konjuguerte-Gradienten-Verfahrens wird das
          Newton-Verfahren verwendet, um die optimale Schrittweite zu bestimmen.
          Zur Überprüfung beider Verfahren wird die Rosenbrock-Funktion
          \begin{equation*}
            f\left(x_1, x_2\right) = \left(1 - x_1\right)^2 + 100 \left(x_2 - x_1^2\right)^2
          \end{equation*}
          mit dem Startpunkt
          \begin{equation*}
            \vec{x_0} = \begin{pmatrix}
              -1 \\
              1
          \end{pmatrix}
          \end{equation*}
          verwendet. Bei beiden Verfahren ergibt sich ein Minimum an der Stelle $(1, 1)^{\text{T}}$. In den Abbildungen \ref{fig:a1_1}  und \ref{fig:a1_2} sind die einzelnen Schritte $\vec{x_n}$ dargestellt.
          \FloatBarrier
          \begin{figure}[H]
              \centering
              \includegraphics[width=0.8\textwidth]{plotgrad_a.pdf}
              \caption{Contour-Plot zur Darstellung der einzelnen Schritte des Gradienten-Verfahrens.}
              \label{fig:a1_1}
          \end{figure}
          \FloatBarrier
          \noindent
          \FloatBarrier
          \begin{figure}[H]
              \centering
              \includegraphics[width=0.8\textwidth]{plotgrad_b.pdf}
              \caption{Contour-Plot zur Darstellung der einzelnen Schritte des Konjugierte-Gradienten-Verfahrens.}
              \label{fig:a1_2}
          \end{figure}
          \FloatBarrier
          \noindent
          Außerdem wird der Fehler $\varepsilon_k = \left| \left|x^k - x^{*} \right| \right|$ in der $L_2$-Norm
          geplottet, was in Abbildung \ref{fig:a1_3} dargestellt ist.
          \FloatBarrier
          \begin{figure}[H]
              \centering
              \includegraphics[width=0.8\textwidth]{err.pdf}
              \caption{Fehler des Gradienten-Verfahrens und des Konjugierte-Gradiente-Verfahrens.}
              \label{fig:a1_3}
          \end{figure}
          \FloatBarrier
          \noindent
\item[b)] Nun wird mit der in Aufgabenteil a) implementierten Methode der konjugierten Gradienten das Minimum
          der Funktion
          \begin{equation*}
            f\left(x_1, x_2\right) = \left(1 + \frac{\exp \left(-10 \left(x_1 x_2 -3\right)^2\right)}{x_1^2 + x_2^2}\right)^{-1}
          \end{equation*}
          bestimmt. Dazu werden die Startwerte
          \begin{equation*}
            x_0 \in \left\{\begin{pmatrix}
              1.5 \\
              2.3
            \end{pmatrix}, \begin{pmatrix}
              -1.7 \\
              -1.9
            \end{pmatrix}, \begin{pmatrix}
              0.5 \\
              0.6
            \end{pmatrix} \right\}
          \end{equation*}
          verwendet. Für die ersten Startwerte ergibt sich das Minimum $\left(\-infty, \infty \right)^{\text{T}}$, für die zweiten das Minimum $\left(0.7161, 0.2992\right)^{\text{T}}$ und für die dritten Startwerte ergibt sich
          $\left(-\text{nan},-\text{nan}\right)^{\text{T}}$.

        Da der Plot bereits in Aufgabenteil a ein wenig seltsam aussieht und auch die Ergebnisse in Aufgabenteil b weniger überzeugend sind. Vermuten wir, dass in der Implementierung des konjugierten Gradientenverfahren ein Fehler ist, den wir leider nicht finden konnten.
\end{itemize}

\section*{Aufgabe 2: BFGS-Verfahren}
\begin{itemize}[leftmargin=*]
\item[a)] Ziel dieser Aufgabe ist die Implementierung des Broyden-Fletcher-Goldfarb-Shanno-Algorithmus
          (BFGS-Algorithmus). Der implementierten Minimierungsfunktion \texttt{BFGS} werden dabei die zu minimierende Funktion \texttt{f}, die Gradientenfunktion \texttt{g}, der Vektor mit den Startwerten \texttt{$x_0$}, die Hessematrix \texttt{$c_0$} und der konstante Wert für die Toletanz \texttt{epsilon}
          übergeben.
\item[b)] Die initiale Hessematrix \texttt{$c_0$} kann über drei Verfahren bestimmt werden, die ebenfalls
          implementiert werden:
          \begin{enumerate}
            \item Eine Möglichkeit ist, die exakte inverse Hessematrix $\tilde{c_0}$ zu berechnen.
            \item Eine weitere Möglichkeit ist die Verwendung einer Diagonalmatrix, die als Diagonalelemente
                  die Diagonalelemente der inversen Hessematrix hat.
            \item Außerdem kann eine Einheitsmatrix verwendet werden, die mit einem Vorfaktor in der typischen
                  Größenordnung von $f(x)$ multipliziert wird. Dazu wird hier $f(\vec{x_0})$ verwendet.
          \end{enumerate}
          Die Implementierung wird nun mit der Rosenbrock-Funktion aus Aufgabe 1 überprüft
          \begin{equation*}
            f\left(x_1, x_2\right) = \left(1 - x_1\right)^2 + 100 \left(x_2 - x_1^2\right)^2 \, .
          \end{equation*}
          Als Startpunkt wird hier ebenfalls
          \begin{equation*}
            \vec{x_0} = \begin{pmatrix}
              -1 \\
              1
          \end{pmatrix}
          \end{equation*}
          verwendet, und die Fehlertoleranz bei $\varepsilon = 10^{-5}$ gesetzt.
          Als Ergbnis der Minimierung des Verfahrens ergibt sich für die verschiedenen Möglichkeiten für die
          initiale Hessematrix in jedem Fall
          \begin{equation*}
            \vec{x} = \begin{pmatrix}
              1 \\
              1
          \end{pmatrix}
          \end{equation*}
\item[c)] Der Vergleich der Verfahren erfolgt über die Iterationszahl $k$.
          Offenbar braucht das Verfahren die wenigsten Iterationsschritte, wenn es mit der exakten inversen
          Hessematrix durchgeführt wird. Das Verfahren mit der Diagonalmatrix und das Verfahren mit der Einheitsmatrix sind gleich aufwändig. Im Verleich mit Aufgabe 1 a) fällt auf, dass das Verfahren
          deutlich schneller ist als das Gradientenverfahren. Ein Vergleich mit dem Konjugierte-Gradienten-
          Verfahren ist aus den oben genannten Gründen leider nicht möglich.
          \FloatBarrier
          \begin{table}[h]
              \centering
          		\label{tab:tab1}
          		\caption{Iterationszahlen des BFGS-Verfahrens für die verschiedenen Möglichkeiten zur Bestimmung der initialen Hessematrix.}
              \begin{tabular}{c c c c}
                  \toprule
          				$ $ & $\text{Exakte Matrix}$ & $\text{Diagonalmatrix}$ & $\text{Einheitsmatrix}$ \\
          				\midrule
                  $k$ & 75 & 84 & 84 \\
                  \bottomrule
              \end{tabular}
          \end{table}
          \FloatBarrier
          \noindent
\end{itemize}
\end{document}
