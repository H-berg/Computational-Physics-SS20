\documentclass[titlepage=firstiscover, captions=tableheading, bibliography=totoc]{scrartcl}
\usepackage[autostyle=true,german=quotes]{csquotes}
\usepackage{scrhack}
\usepackage{enumitem}
\usepackage{caption}
\usepackage[aux]{rerunfilecheck}
\usepackage{subcaption}
\usepackage{fontspec}
\usepackage[dvips]{graphicx}
\usepackage{floatflt,epsfig}

\usepackage{polyglossia}
\setmainlanguage{german}

\usepackage[unicode]{hyperref}
\usepackage{bookmark}
\title{Computational Physics}
\subtitle{Übungsblatt 7}
\author{
Miriam Simm\\
\texorpdfstring{\href{mailto:miriam.simm@tu-dortmund.de}{miriam.simm@tu-dortmund.de}\and}{,}
Katrin Bolsmann\\
\texorpdfstring{\href{mailto:katrin.bolsmann@tu-dortmund.de}{katrin.bolsmann@tu-dortmund.de}}{,}\\
\\
Mario Alex Hollberg\\
\texorpdfstring{\href{mailto:mario-alex.hollberg@tu-dortmund.de}{mario-alex.hollberg@tu-dortmund.de}}{}
}
\date{Abgabe: 15. Mai 2020}
\usepackage{amsmath}
\usepackage{amssymb}
\usepackage{mathtools}
\usepackage[
    math-style=ISO,
    bold-style=ISO,
    sans-style=italic,
    nabla=upright,
    partial=upright,
]{unicode-math}

\setmathfont{Latin Modern Math}

\usepackage[
  locale=DE,
  separate-uncertainty=true,
  per-mode=symbol-or-fraction,
]{siunitx}

\usepackage{multicol}
\setlength{\columnsep}{1pt} %space between columns

\usepackage{booktabs}
\usepackage[x11names, table]{xcolor}
\usepackage{graphicx}
\usepackage{grffile}
\usepackage{xfrac}
\usepackage{xcolor}

\usepackage{float}
\floatplacement{figure}{h}
\floatplacement{table}{h}
\usepackage[
  section,
  below,
]{placeins}

\usepackage{expl3}
\usepackage{xparse}
\ExplSyntaxOn
\NewDocumentCommand \E {} {\symup{e}}
\ExplSyntaxOff

% Literaturverzeichnis
\usepackage[
  backend=biber,
]{biblatex}
% Quellendatenbank
\addbibresource{literatur.bib}

\usepackage[
  version=4,
  math-greek=default,
  text-greek=default,
]{mhchem}


\raggedcolumns


\begin{document}

\maketitle

\section*{Aufgabe 1: Eindimensionale Integration}
Ziel der Aufgabe ist die numerische Berechung der folgenden Integrale mit einer der Integrationsroutinen,
die bereits auf dem letzten Übungsblatt implementiert wurden:
\begin{align*}
	I_1 &= \dashint_{-1}^{1} \text{d} t \,\frac{\text{e}^{t}}{t} \\
	I_2 &= \int_0^{\infty} \text{d} t \, \frac{\text{e}^{-t}}{\sqrt{t}} \\
  I_3 &= \int_{-\infty}^{\infty} \text{d} t \, \frac{\sin (t)}{t}
\end{align*}
\noindent

\begin{itemize}[leftmargin=*]
\item[a)] Da es sich bei diesem Integral um ein Hauptwertintegral handelt, sollte nicht direkt integriert werden. Stattdessen wird das Integral wie folgt aufgeteilt:
          \begin{align*}
          I_1 &= \dashint_{-1}^{1} \text{d} t \,\frac{\text{e}^{t}}{t} = \underbrace{\int_{-1}^{-\increment} \, \frac{\text{e}^{t}}{t} \, \text{d} t + \int_{\increment}^{1} \, \frac{\text{e}^{t}}{t} \, \text{d} t}_{\text{Numerische Berechnung}} + \underbrace{\int_{-\increment}^{\increment} \, \frac{\text{e}^{t}}{t} \, \text{d} t}_{I'} \\
          I' &= \lim_{\varepsilon\to 0} \left(\int_{-\increment}^{-\varepsilon} \frac{\text{e}^{t}}{t}  \, \text{d} t +  \underbrace{\int_{\varepsilon}^{\increment} \frac{\text{e}^{t}}{t}  \, \text{d} t}_{I''}\right) \\
             &= \lim_{\varepsilon\to 0} \left(\int_{-\increment}^{-\varepsilon} \frac{\text{e}^{t}}{t}  \, \text{d} t + \int_{-\varepsilon}^{-\increment} \frac{\text{e}^{-t}}{(-t)}  \, \text{d} (-t)\right) \\
             &= \lim_{\varepsilon\to 0} \underbrace{\left(\int_{-\increment}^{-\varepsilon} \frac{\text{e}^{t}-\text{e}^{-t}}{t}  \, \text{d} t \right)}_{\text{hebbare Singularität}} \\
             &= \int_{-\increment}^{0} \frac{\text{e}^{t}-\text{e}^{-t}}{t}  \, \text{d} t \\
          \intertext{mit}
          \lim_{t \to 0} \frac{\text{e}^{t}-\text{e}^{-t}}{t} &= \lim_{t \to 0} \frac{\text{e}^{t}+\text{e}^{-t}}{1} = 2
          \end{align*}
\item[b)] Bei diesem Integral wird bis zu einer oberen Grenze $x_\text{max} = 100000$ ingetriert,
          \begin{align*}
          I_2 &= \int_0^{\infty} \text{d} t \, \frac{\text{e}^{-t}}{\sqrt{t}} \\
              &=  \underbrace{\int_0^{x_\text{max}} \text{d} t \, \frac{\text{e}^{-t}}{\sqrt{t}}}_{I'} +  \underbrace{\int_{x_\text{max}}^{\infty} \text{d} t \, \frac{\text{e}^{-t}}{\sqrt{t}}}_{\text{Fehler}} \\
          I' &= \int_0^{x_\text{max}} \text{d} t \, \frac{\text{e}^{-t}}{\sqrt{t}} \\
             &= \int_0^{\increment x} \text{d} t \, \frac{\text{e}^{-t}}{\sqrt{t}}
          \end{align*}
          wobei die erste Integration für ein sehr kleines $\increment x$ vernachlässigt werden kann.
\item[c)] Bei diesem Integral wird ausgenutzt, dass die zu integrierende Funktion symmetrisch ist:
          \begin{align*}
          I_3 &= \int_{-\infty}^{\infty} \text{d} t \, \frac{\sin (t)}{t} \\
              &= 2 \int_{0}^{\infty} \text{d} t \, \frac{\sin (t)}{t} \\
              &= 2 \int_{0}^{1} \text{d} t \, \frac{\sin (t)}{t} + 2 \underbrace{\int_{1}^{\infty} \text{d} t \, \frac{\sin (t)}{t}}_{I'} \\
          \intertext{Das erste Integral hat bei $t=0$ eine hebbare Singularität. Für $I'$ wird eine Substition mit $t = \frac{1}{x}$ durchgeführt.}
          I' &= \int_{1}^{\infty} \text{d} t \, \frac{\sin (t)}{t} \\
             &= \int_{1}^{0}  \left(- \frac{\text{d} x}{x^2}\right) \, \frac{\sin (\frac{1}{x})}{\frac{1}{x}} \\
             &= \int_{1}^{0} \text{d} x \; \frac{\sin (\frac{1}{x})}{x}
          \end{align*}
\end{itemize}
Mit diesen Umformungen wird die Integration nun implementiert. Die Ergebnisse unter Angabe der verwendeten Integrationsmethode
befinden sich in der folgenden Tabelle.
\FloatBarrier
\begin{table}[h]
    \centering
		\label{tab:tab1}
		\caption{Ergebnisse der Berechnung der einzelen Integrale.}
    \begin{tabular}{c c c}
        \toprule
				$$ & $\text{Ergbnis}$ & $\text{Methode}$ \\
				\midrule
				$I_1$ & 2.1132 & $\text{Simpsonregel}$ \\
				$I_2$ & 1.77034 & $\text{Mittelpunktsregel}$\\
        $I_3$ & 3.14159 & $\text{Mittelpunktsregel}$\\
        \bottomrule
    \end{tabular}
\end{table}
\FloatBarrier
\noindent
Aufgabenteil c) wird außerdem analytisch berechnet. Das Integral hat, wie oben bereits erwähnt, eine hebbare Singularität
bei $t = 0$, wozu die Regel von de l'Hospital angewandt wird:
\begin{equation*}
  \lim_{t \to 0} \frac{\sin (t)}{t} \stackrel{"\frac{0}{0}"}{=} \lim_{t \to 0} \cos (t) = 1
\end{equation*}
Der Sinus dargestellt durch
\begin{equation*}
  \sin (t) = \frac{1}{2i} \left(\symup{e}^{it}-\symup{e}^{-it}\right)
\end{equation*}
Zur Lösung des Integrals wird nun der Cauchy-Hauptwert verwendet und das Integral $\int_{-\infty}^{\infty} \frac{\text{e}^{it}}{t} \, \text{d} t$
betrachtet:
\begin{align*}
  \text{CH} \int_{-\infty}^{\infty} f(x) \text{d} x &= \lim_{r \to 0} \left(\int_{-r}^{0}f(x) \text{d} x  + \int_{0}^{r}f(x) \text{d} x\right) \\
  \text{CH} \int_{-\infty}^{\infty} \frac{\text{e}^{it}}{t} \, \text{d} t &= \lim_{R \to \infty, \varepsilon \to 0} \left(\int_{-R}^{-\varepsilon} \frac{\text{e}^{i \omega t}}{t} \, \text{d} t + \int_{\varepsilon}^{R} \frac{\text{e}^{i \omega t}}{t} \, \text{d} t \right) \\
  &= \begin{cases}
  i \pi   & \text{für} \; \omega > 0 \\
  -i \pi  & \text{für} \; \omega < 0
      \end{cases}
\end{align*}
Diese Ergebnisse werden nun für das hier vorliende Integral verwendet.
\begin{align*}
  \text{CH} \int_{-\infty}^{\infty} \frac{\sin (t)}{t} \text{d} t &= \lim_{R \to \infty, \varepsilon \to 0} \frac{1}{2i} \left(\int_{-R}^{-\varepsilon} \frac{\text{e}^{i t}}{t} \, \text{d} t + \int_{\varepsilon}^{R} \frac{\text{e}^{i t}}{t} \, \text{d} t - \int_{-R}^{-\varepsilon} \frac{\text{e}^{-i t}}{t} \, \text{d} t - \int_{\varepsilon}^{R} \frac{\text{e}^{-i t}}{t} \, \text{d} t \right) \\
  &= \frac{1}{2 i} \left(i \pi - \left(-i \pi\right)\right) \\
  &= \pi
\end{align*}
Dieses Ergebnis stimmt, unter Betrachtung eines gewissen Fehlers, mit dem Resultat aus der numerischen Berechnung in \ref{tab:tab1} überein.
\section*{Aufgabe 2: Mehrdimensionale Integration in der Elektrostatik}
Das elektrostatische Potential entlang der x-Achse:

\begin{equation*}
  \phi(x)=\frac{1}{4 \pi \epsilon_{0}} \int \mathrm{d} x^{\prime} \int \mathrm{d} y^{\prime} \int \mathrm{d} z^{\prime} \frac{\rho\left(x^{\prime}, y^{\prime}, z^{\prime}\right)}{\left[\left(x-x^{\prime}\right)^{2}+y^{\prime 2}+z^{\prime 2}\right]^{1 / 2}}
\end{equation*}

\noindent
für zwei Ladungsverteilungen in einem Würfel der Kantenlänge $2a$ ist zu bestimmen.


\subsection*{a)}

\begin{equation*}
  \rho(x, y, z)=\left\{\begin{array}{ll}\rho_{0}, & |x|<a,|y|<a,|z|<a \\ 0, & \text { sonst }\end{array}\right.
\end{equation*}


\subsubsection*{(1)}

Das Integral wird folgendermaßen einheitenlos gemacht. Dabei wird $a$ gleich Eins
gesetzt.

\begin{equation*}
  x' \rightarrow \frac{x}{a} \quad
  \phi' \rightarrow \phi\frac{4\pi\epsilon_0}{\rho_0 a^2  }
\end{equation*}

\subsubsection*{(2) +(3)}
Das Integral wird mit der Mittelpunktsregel außerhalb des Würfels für x-Werte
$x / a=0.1 n$ mit $n \in\{11,12, \ldots, 80\}$ ausgewertet. Die Asymptotik des
Potentials für große x-Werte wird mit einer Multipolentwicklung bis zur ersten
nicht-verschwindenden Ordnung abgeschätzt:

\begin{equation*}
  \frac{1}{\sqrt{x^2+y^2+z^2}} \int_{-1}^1 \int_{-1}^1 \int_{-1}^1 dx'dy'dz'
  \overset{\mathrm{y=z=0}}{=} \frac{8}{x}
\end{equation*}

und ist zusammen mit der Integral-Auswertung in Abbildung \ref{fig:2a.2} abgebildet.

\begin{figure}[H]
    \centering
    \includegraphics[width=0.8\textwidth]{Abbildungen/out_a.pdf}
    \caption{x-Werte außerhalb des Potentials}
    \label{fig:2a.2}
\end{figure}

\subsubsection*{(4)}
Nun wird das Potential für x-Wert innerhalb des Würfels:
$x / a=0.1 n$ mit $n \in\{0,1, \ldots, 10\}$ ausgewertet und in Abbildung \ref{fig:2a.4}
dargestellt. Singularitäten können hier zu Probleme führen, weshalb die Schrittweite $h$
in allen drei Integrationen auf eine irrationale Zahl $\frac{1}{30\pi}$ gesetzt wird.
Es ist nun sehr unwahrscheinlich, dass $x$ gleich $x'$, und somit $(x-x')^2$
(und damit möglicherweise auch der ganze Nenner) Null wird.

\begin{figure}[H]
    \centering
    \includegraphics[width=0.8\textwidth]{Abbildungen/in_a.pdf}
    \caption{x-Werte innerhalb des Potentials}
    \label{fig:2a.4}
\end{figure}

\subsection*{b)}

\begin{equation}\rho(x, y, z)=\left\{\begin{array}{ll}
\rho_{0} \frac{x}{a}, & |x|<a,|y|<a,|z|<a \\
0, & \text { sonst }
\end{array}\right.\end{equation}

\subsubsection*{(1)}
Analog zu a)

\subsubsection*{(2)+(3)}
Multipolentwicklung bis zu ersten nicht-verschwindenden Ordnung:

\begin{equation*}
  \frac{1}{\sqrt{x^2+y^2+z^2}} \int_{-1}^1 \int_{-1}^1 \int_{-1}^1 x^2 dx'dy'dz'
  \overset{\mathrm{y=z=0}}{=} \frac{8}{3x^2}
\end{equation*}

\begin{figure}[H]
    \centering
    \includegraphics[width=0.8\textwidth]{Abbildungen/out_b.pdf}
    \caption{x-Werte außerhalb des Potentials}
    \label{fig:2b.2}
\end{figure}

\subsubsection*{(4)}
Analog zu a)

\begin{figure}[H]
    \centering
    \includegraphics[width=0.8\textwidth]{Abbildungen/in_b.pdf}
    \caption{x-Werte innerhalb des Potentials}
    \label{fig:2b.4}
\end{figure}

\end{document}
