\documentclass[titlepage=firstiscover, captions=tableheading, bibliography=totoc]{scrartcl}
\usepackage[autostyle=true,german=quotes]{csquotes}
\usepackage{scrhack}
\usepackage{enumitem}
\usepackage{caption}
\usepackage[aux]{rerunfilecheck}
\usepackage{subcaption}
\usepackage{fontspec}
\usepackage[dvips]{graphicx}
\usepackage{floatflt,epsfig}

\usepackage{polyglossia}
\setmainlanguage{german}

\usepackage[unicode]{hyperref}
\usepackage{bookmark}
\title{Computational Physics}
\subtitle{Übungsblatt 4}
\author{
Miriam Simm\\
\texorpdfstring{\href{mailto:miriam.simm@tu-dortmund.de}{miriam.simm@tu-dortmund.de}\and}{,}
Katrin Bolsmann\\
\texorpdfstring{\href{mailto:katrin.bolsmann@tu-dortmund.de}{katrin.bolsmann@tu-dortmund.de}}{,}\\
\\
Mario Alex Hollberg\\
\texorpdfstring{\href{mailto:mario-alex.hollberg@tu-dortmund.de}{mario-alex.hollberg@tu-dortmund.de}}{}
}
\date{Abgabe: 22. Mai 2020}
\usepackage{amsmath}
\usepackage{amssymb}
\usepackage{mathtools}
\usepackage[
    math-style=ISO,
    bold-style=ISO,
    sans-style=italic,
    nabla=upright,
    partial=upright,
]{unicode-math}

\setmathfont{Latin Modern Math}

\usepackage[
  locale=DE,
  separate-uncertainty=true,
  per-mode=symbol-or-fraction,
]{siunitx}

\usepackage{multicol}
\setlength{\columnsep}{1pt} %space between columns

\usepackage{booktabs}
\usepackage[x11names, table]{xcolor}
\usepackage{graphicx}
\usepackage{grffile}
\usepackage{xfrac}
\usepackage{xcolor}

\usepackage{float}
\floatplacement{figure}{h}
\floatplacement{table}{h}
\usepackage[
  section,
  below,
]{placeins}

\usepackage{expl3}
\usepackage{xparse}
\ExplSyntaxOn
\NewDocumentCommand \E {} {\symup{e}}
\ExplSyntaxOff

% Literaturverzeichnis
\usepackage[
  backend=biber,
]{biblatex}
% Quellendatenbank
\addbibresource{literatur.bib}

\usepackage[
  version=4,
  math-greek=default,
  text-greek=default,
]{mhchem}

\def\Xint#1{\mathchoice
   {\XXint\displaystyle\textstyle{#1}}%
   {\XXint\textstyle\scriptstyle{#1}}%
   {\XXint\scriptstyle\scriptscriptstyle{#1}}%
   {\XXint\scriptscriptstyle\scriptscriptstyle{#1}}%
   \!\int}
\def\XXint#1#2#3{{\setbox0=\hbox{$#1{#2#3}{\int}$}
     \vcenter{\hbox{$#2#3$}}\kern-.5\wd0}}
\def\ddashint{\Xint=}
\def\dashint{\Xint-}


\raggedcolumns


\begin{document}

\maketitle

\section*{Aufgabe 1: Eindimensionale Integration}

\section*{Aufgabe 2: Mehrdimensionale Integration in der Elktrostatik}

Das elektrostatische Potential entlang der x-Achse:

\begin{equation*}
  \phi(x)=\frac{1}{4 \pi \epsilon_{0}} \int \mathrm{d} x^{\prime} \int \mathrm{d} y^{\prime} \int \mathrm{d} z^{\prime} \frac{\rho\left(x^{\prime}, y^{\prime}, z^{\prime}\right)}{\left[\left(x-x^{\prime}\right)^{2}+y^{\prime 2}+z^{\prime 2}\right]^{1 / 2}}
\end{equation*}

\noindent
für zwei Ladungsverteilungen in einem Würfel der Kantenlänge $2a$ ist zu bestimmen.


\subsection*{a)}

\begin{equation*}
  \rho(x, y, z)=\left\{\begin{array}{ll}\rho_{0}, & |x|<a,|y|<a,|z|<a \\ 0, & \text { sonst }\end{array}\right.
\end{equation*}


\subsubsection*{(1)}

Das Integral wird folgendermaßen einheitenlos gemacht. Dabei wird $a$ gleich Eins
gesetzt.

\begin{equation*}
  x' \rightarrow \frac{x}{a} \quad
  \phi' \rightarrow \phi\frac{4\pi\epsilon_0}{\rho_0 a^2  }
\end{equation*}

\subsubsection*{(2) +(3)}
Das Integral wird mit der Mittelpunktsregel außerhalb des Würfels für x-Werte
$x / a=0.1 n$ mit $n \in\{11,12, \ldots, 80\}$ ausgewertet. Die Asymptotik des
Potentials für große x-Werte wird mit einer Multipolentwicklung bis zur ersten
nicht-verschwindenden Ordnung abgeschätzt:

\begin{equation*}
  \frac{1}{\sqrt{x^2+y^2+z^2}} \int_{-1}^1 \int_{-1}^1 \int_{-1}^1 dx'dy'dz'
  \overset{\mathrm{y=z=0}}{=} \frac{8}{x}
\end{equation*}

und ist zusammen mit der Integral-Auswertung in Abbildung \ref{fig:2a.2} abgebildet.

\begin{figure}[H]
    \centering
    \includegraphics[width=0.8\textwidth]{Abbildungen/out_a.pdf}
    \caption{x-Werte außerhalb des Potentials}
    \label{fig:2a.2}
\end{figure}

\subsubsection*{(4)}
Nun wird das Potential für x-Wert innerhalb des Würfels:
$x / a=0.1 n$ mit $n \in\{0,1, \ldots, 10\}$ ausgewertet und in Abbildung \ref{fig:2a.4}
dargestellt. Singularitäten können hier zu Probleme führen, weshalb die Schrittweite $h$
in allen drei Integrationen auf eine irrationale Zahl $\frac{1}{30\pi}$ gesetzt wird.
Es ist nun sehr unwahrscheinlich, dass $x$ gleich $x'$, und somit $(x-x')^2$
(und damit möglicherweise auch der ganze Nenner) Null wird.

\begin{figure}[H]
    \centering
    \includegraphics[width=0.8\textwidth]{Abbildungen/in_a.pdf}
    \caption{x-Werte innerhalb des Potentials}
    \label{fig:2a.4}
\end{figure}

\subsection*{b)}

\begin{equation}\rho(x, y, z)=\left\{\begin{array}{ll}
\rho_{0} \frac{x}{a}, & |x|<a,|y|<a,|z|<a \\
0, & \text { sonst }
\end{array}\right.\end{equation}

\subsubsection*{(1)}
Analog zu a)

\subsubsection*{(2)+(3)}
Multipolentwicklung bis zu ersten nicht-verschwindenden Ordnung:

\begin{equation*}
  \frac{1}{\sqrt{x^2+y^2+z^2}} \int_{-1}^1 \int_{-1}^1 \int_{-1}^1 x^2 dx'dy'dz'
  \overset{\mathrm{y=z=0}}{=} \frac{8}{3x^2}
\end{equation*}

\begin{figure}[H]
    \centering
    \includegraphics[width=0.8\textwidth]{Abbildungen/out_b.pdf}
    \caption{x-Werte außerhalb des Potentials}
    \label{fig:2b.2}
\end{figure}

\subsubsection*{(4)}
Analog zu a)

\begin{figure}[H]
    \centering
    \includegraphics[width=0.8\textwidth]{Abbildungen/in_b.pdf}
    \caption{x-Werte innerhalb des Potentials}
    \label{fig:2b.4}
\end{figure}

\end{document}
