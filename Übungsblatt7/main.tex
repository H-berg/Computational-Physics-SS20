\documentclass[titlepage=firstiscover, captions=tableheading, bibliography=totoc]{scrartcl}
\usepackage[autostyle=true,german=quotes]{csquotes}
\usepackage{scrhack}
\usepackage{enumitem}
\usepackage{caption}
\usepackage[aux]{rerunfilecheck}
\usepackage{subcaption}
\usepackage{fontspec}
\usepackage[dvips]{graphicx}
\usepackage{floatflt,epsfig}

\usepackage{polyglossia}
\setmainlanguage{german}

\usepackage[unicode]{hyperref}
\usepackage{bookmark}
\title{Computational Physics}
\subtitle{Übungsblatt 4}
\author{
Miriam Simm\\
\texorpdfstring{\href{mailto:miriam.simm@tu-dortmund.de}{miriam.simm@tu-dortmund.de}\and}{,}
Katrin Bolsmann\\
\texorpdfstring{\href{mailto:katrin.bolsmann@tu-dortmund.de}{katrin.bolsmann@tu-dortmund.de}}{,}\\
\\
Mario Alex Hollberg\\
\texorpdfstring{\href{mailto:mario-alex.hollberg@tu-dortmund.de}{mario-alex.hollberg@tu-dortmund.de}}{}
}
\date{Abgabe: 22. Mai 2020}
\usepackage{amsmath}
\usepackage{amssymb}
\usepackage{mathtools}
\usepackage[
    math-style=ISO,
    bold-style=ISO,
    sans-style=italic,
    nabla=upright,
    partial=upright,
]{unicode-math}

\setmathfont{Latin Modern Math}

\usepackage[
  locale=DE,
  separate-uncertainty=true,
  per-mode=symbol-or-fraction,
]{siunitx}

\usepackage{multicol}
\setlength{\columnsep}{1pt} %space between columns

\usepackage{booktabs}
\usepackage[x11names, table]{xcolor}
\usepackage{graphicx}
\usepackage{grffile}
\usepackage{xfrac}
\usepackage{xcolor}

\usepackage{float}
\floatplacement{figure}{h}
\floatplacement{table}{h}
\usepackage[
  section,
  below,
]{placeins}

\usepackage{expl3}
\usepackage{xparse}
\ExplSyntaxOn
\NewDocumentCommand \E {} {\symup{e}}
\ExplSyntaxOff

% Literaturverzeichnis
\usepackage[
  backend=biber,
]{biblatex}
% Quellendatenbank
\addbibresource{literatur.bib}

\usepackage[
  version=4,
  math-greek=default,
  text-greek=default,
]{mhchem}

\def\Xint#1{\mathchoice
   {\XXint\displaystyle\textstyle{#1}}%
   {\XXint\textstyle\scriptstyle{#1}}%
   {\XXint\scriptstyle\scriptscriptstyle{#1}}%
   {\XXint\scriptscriptstyle\scriptscriptstyle{#1}}%
   \!\int}
\def\XXint#1#2#3{{\setbox0=\hbox{$#1{#2#3}{\int}$}
     \vcenter{\hbox{$#2#3$}}\kern-.5\wd0}}
\def\ddashint{\Xint=}
\def\dashint{\Xint-}


\raggedcolumns


\begin{document}

\maketitle

\section*{Aufgabe 1: Runge-Kutta-Verfahren}
\subsection*{a)}

Das RK-Verfahren 4.Ordnung wird auf die Newtonsche Bewegungsgleichung für ein Teilchen implementiert und an einem harmonischen Oszillator mit zwei unterschiedlichen Anfangsbedingungen getest:\\
1.Test:\\
$\vec{r}(0)$ beliebig und $\vec{v}(0)=\vec{0} \implies \vec{r}(0) = (1,2,3)^T$ und $\vec{v}(0) = (1,1,1)^T$ gewählt\\
Das Teilchen wird praktisch an einem Punkt losgelassen und fängt an harmonisch
zu pendeln.

\begin{figure}[H]
    \centering
    \includegraphics[width=0.8\textwidth]{Abbildungen/A1_1a_1.pdf}
    \caption{Pendel-Trajektorie, wobei die Masse des Teilchens 1 beträgt}
    \label{fig:A1_1a_1}
\end{figure}

\noindent
2.Test:\\
$\vec{v}(0) \neq \vec{0}$ und $\vec{v}(0) \nparallel \vec{r}(0)  \implies \vec{r}(0) = (1,2,3)^T$ und $\vec{v}(0) = (1,1,1)^T$ gewählt\\
Da das Teilchen dieses Mal einen Anfangsimpuls besitzt, bewegt es sich auf einer
Ellipse.

\begin{figure}[H]
    \centering
    \includegraphics[width=0.8\textwidth]{Abbildungen/A1_1a_2.pdf}
    \caption{Elliptische Trajektorie, wobei die Masse des Teilchens 1 beträgt}
    \label{fig:A1_1a_2}
\end{figure}
\noindent
Da in beiden Fällen keine Dämpfung vorliegt, ist die Trajektorie immer die gleiche.
Somit werden ältere Wegpunkten von Neueren überlagert (weshalb man keine dunkle Punkte mehr sehen kann).

\subsection*{b)}
Damit die Schrittweite $h$ der Toleranzgrenze von $\left|\vec{r}_{0}-\vec{r}_{i}\right|<10^{-5}$ bei $i=10$ Schwingungen genügt, muss $h$ in der Größenornung von $10^{-8}$ liegen.

\subsection*{c)}
Als nächstes wird die Energieerhaltung des 1.Tests aus Aufgabenteil a) überprüft.
Für einen harmonischen Oszillator gilt:
\begin{equation}
  E_{\text{ges}} = E_{\text{pot}} + E_{\text{kin}}
  = \frac{1}{2} m \omega^{2} \hat{x}^{2} + \frac{1}{2} m \hat{v}^{2}
  = const.
\end{equation}
\noindent
In Abbildung \ref{fig:A1_1c} ist die relative Gesamtenergie zu jedem Zeitschritt aufgetragen.
Um die Änderung der Gesamtenergie deutlicher zu machen, wird jeder Wert
von dem Anfangswert $E_{\text{ges}}(t=0)$ abgezogen. Es ist deutlich eine sinus förmige Bewegung zu erkennen, welche nicht für eine Energieerhaltung spricht.
Vermutlich wurde das RK-Verfahren nicht sauber genug implementiert.

\begin{figure}[H]
    \centering
    \includegraphics[width=0.8\textwidth]{Abbildungen/A1_1c.pdf}
    \caption{Energieerhaltung eines Teilchens mit eier Masse von eins}
    \label{fig:A1_1c}
\end{figure}

\section*{Aufgabe 2: Adams-Bashforth-Verfahren}
\subsection*{a)}
Das Adams-Bashforth-Verfahren für die Bewegungsgleichung

\begin{equation*}
  \ddot{x}=-x-\alpha \dot{x}
\end{equation*}
\noindent und verschiedene Fälle für $\alpha$ werden untersucht. Dabei werden
zunächst vier Anfangspunkte mittels der in Aufgabe 1 implementierten RK-Methode
bestimmt.\\
Für den Fall das $\alpha > 0$ ist, zeigt sich in Abbildung \ref{fig:A2_a1}, wie zu erwarten, ein gedämpfter harmonischer Oszillator. Das Teilchen schwingt mit der Zeit in einer immer enger werdenden Kreisbahn.\\
Ist $\alpha = 0$ liegt ein ungedämpfter harmonischer Oszillator wie in Aufgabe 1 vor. Das Teilchen bewegt sich also immer auf der gleichen Kreisbahn (siehe Abbildung \ref{fig:A2_a2}).\\
Zuletzt wird $\alpha < 0$ betrachtet, in der eine erzwungende harmonische Oszillation zu erwarten ist. Diese zeigt sich für den Fall $\alpha = -0.1$ in
Abbildung \ref{fig:A2_a3} gut. Das Teilchen bewegt sich auf einer immer größer werdenden Kreisbahn.\\
Die mittels dem RK-Verfahren ermittelten vier Startwerte stechen in den jeweiligen Abbildungen besonders hervor.

\begin{figure}[H]
    \centering
    \includegraphics[width=0.8\textwidth]{Abbildungen/A2_a1.pdf}
    \caption{Gedämpfer harm. Osz. mit $\alpha = 0.1$}
    \label{fig:A2_a1}
\end{figure}

\begin{figure}[H]
    \centering
    \includegraphics[width=0.8\textwidth]{Abbildungen/A2_a2.pdf}
    \caption{Harm. Osz. mit $\alpha = 0$}
    \label{fig:A2_a2}
\end{figure}

\begin{figure}[H]
    \centering
    \includegraphics[width=0.8\textwidth]{Abbildungen/A2_a3.pdf}
    \caption{Erzwungener harm. Osz. mit $\alpha = -0.1$}
    \label{fig:A2_a3}
\end{figure}


\subsection*{b)}
Als nächstes wird die Energieerhaltung für $\alpha = 0.1$ für $t=20$ Zeitschritten untersucht. Da keine Reibungsenergie berücksichtigt wird,
sinkt die Gesamtenergie des gedämpften harmonischen Oszillators, wie in Abbildung \ref{fig:A2_b} zu sehen ist, mit der Zeit.

\begin{figure}[H]
    \centering
    \includegraphics[width=0.8\textwidth]{Abbildungen/A2_b.pdf}
    \caption{Gedämpfer harm. Osz. mit $\alpha = 0.1$}
    \label{fig:A2_b}
\end{figure}


\end{document}
