\documentclass[titlepage=firstiscover, captions=tableheading, bibliography=totoc]{scrartcl}
\usepackage[autostyle=true,german=quotes]{csquotes}
\usepackage{scrhack}
\usepackage{enumitem}
\usepackage{caption}
\usepackage[aux]{rerunfilecheck}
\usepackage{subcaption}
\usepackage{fontspec}
\usepackage[dvips]{graphicx}
\usepackage{floatflt,epsfig}

\usepackage{polyglossia}
\setmainlanguage{german}

\usepackage[unicode]{hyperref}
\usepackage{bookmark}
\title{Computational Physics}
\subtitle{Übungsblatt 4}
\author{
Miriam Simm\\
\texorpdfstring{\href{mailto:miriam.simm@tu-dortmund.de}{miriam.simm@tu-dortmund.de}\and}{,}
Katrin Bolsmann\\
\texorpdfstring{\href{mailto:katrin.bolsmann@tu-dortmund.de}{katrin.bolsmann@tu-dortmund.de}}{,}\\
\\
Mario Alex Hollberg\\
\texorpdfstring{\href{mailto:mario-alex.hollberg@tu-dortmund.de}{mario-alex.hollberg@tu-dortmund.de}}{}
}
\date{Abgabe: 22. Mai 2020}
\usepackage{amsmath}
\usepackage{amssymb}
\usepackage{mathtools}
\usepackage[
    math-style=ISO,
    bold-style=ISO,
    sans-style=italic,
    nabla=upright,
    partial=upright,
]{unicode-math}

\setmathfont{Latin Modern Math}

\usepackage[
  locale=DE,
  separate-uncertainty=true,
  per-mode=symbol-or-fraction,
]{siunitx}

\usepackage{multicol}
\setlength{\columnsep}{1pt} %space between columns

\usepackage{booktabs}
\usepackage[x11names, table]{xcolor}
\usepackage{graphicx}
\usepackage{grffile}
\usepackage{xfrac}
\usepackage{xcolor}

\usepackage{float}
\floatplacement{figure}{h}
\floatplacement{table}{h}
\usepackage[
  section,
  below,
]{placeins}

\usepackage{expl3}
\usepackage{xparse}
\ExplSyntaxOn
\NewDocumentCommand \E {} {\symup{e}}
\ExplSyntaxOff

% Literaturverzeichnis
\usepackage[
  backend=biber,
]{biblatex}
% Quellendatenbank
\addbibresource{literatur.bib}

\usepackage[
  version=4,
  math-greek=default,
  text-greek=default,
]{mhchem}

\def\Xint#1{\mathchoice
   {\XXint\displaystyle\textstyle{#1}}%
   {\XXint\textstyle\scriptstyle{#1}}%
   {\XXint\scriptstyle\scriptscriptstyle{#1}}%
   {\XXint\scriptscriptstyle\scriptscriptstyle{#1}}%
   \!\int}
\def\XXint#1#2#3{{\setbox0=\hbox{$#1{#2#3}{\int}$}
     \vcenter{\hbox{$#2#3$}}\kern-.5\wd0}}
\def\ddashint{\Xint=}
\def\dashint{\Xint-}


\raggedcolumns


\begin{document}

\maketitle

\section*{Monte-Carlo-Simulation eines einzelnen Spins}
Bei dieser Aufgabe wird eine MC-Simulation mittels des Metropolis-Alogrithmus
eines einzelnen Spins $\sigma$ = $\pm$1 mit der Energie
\begin{equation*}
  \mathcal{H}=-\sigma H
\end{equation*}
\noindent
im äußeren Magnetfeld $H$ implementiert.\\
\\
Folgende Schritte werden dabei gemacht:

\begin{enumerate}
  \item Der Spin $\sigma$ wird auf +1 gesetzt. Alternativ könnte man hier auch
  den Anfangsspin zufällig wählen.
  \item Die Energiedifferenz $\Delta E = \Delta\sigma H$ mit $\Delta\sigma = \pm$2
  und die Übergangswahrscheinlichkeit $p = \exp{\bigl(-\beta\Delta E\bigr)}$ werden bestimmt.
    \begin{itemize}
      \item Falls $\Delta E \leq$ 0, dann wird der Spin-Flip akzeptiert, da dieser
      Zustand energetisch günstiger ist.
      \item Falls $\Delta E >$ 0, dann wird die Übergangswahrscheinlichkeit $p$
      mit einer gleichverteilten Vorschlagswarhscheinlichkeit $V \in [0,1]$
      verglichen. Ist $V \leq p$, so wird der Spin-Flip akzeptiert. Ansonsten
      wird der Zustand beigehalten.
    \end{itemize}
  \item Die Magnetisierung $m$ wird aktualisiert.
  \item Wiederholung der Schritte (2) bis (3).
\end{enumerate}
\noindent
Die numerisch bestimmte Magnetisierung $m$ wird zuletzt auf die betragsmäßig
größten Magnetisierung $m_{\text{max}}$ normiert. In Abbildung \ref{fig:tanh}
wird das numerische und das analytische Ergebnis:
$m = \tanh{\bigl(\beta H \bigr)}$ mit
$\beta = \frac{1}{k_{\text{b}}T}$ = 1 dargestellt.

\begin{figure}[H]
    \centering
    \includegraphics[width=0.8\textwidth]{Abbildungen/tanh.pdf}
    \caption{Vergleich zwischen dem analytischen und dem numerischen Ergebnis für
    die Magnetisierung $m$ eines einzelnen Spins. Der Metropolis-Algorithmus
    wird mit $\SI{e4}{}$ Werten für das äußere Magnetfeld $H$, mit jeweils
    $\SI{e6}{}$ Schritten durchgeführt.}
    \label{fig:tanh}
\end{figure}


\end{document}
