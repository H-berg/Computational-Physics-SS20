\documentclass[titlepage=firstiscover, captions=tableheading, bibliography=totoc]{scrartcl}
\usepackage[autostyle=true,german=quotes]{csquotes}
\usepackage{scrhack}
\usepackage{enumitem}
\usepackage{caption}
\usepackage[aux]{rerunfilecheck}
\usepackage{subcaption}
\usepackage{fontspec}
\usepackage[dvips]{graphicx}
\usepackage{floatflt,epsfig}

\usepackage{polyglossia}
\setmainlanguage{german}

\usepackage[unicode]{hyperref}
\usepackage{bookmark}
\title{Computational Physics}
\subtitle{Übungsblatt 4}
\author{
Miriam Simm\\
\texorpdfstring{\href{mailto:miriam.simm@tu-dortmund.de}{miriam.simm@tu-dortmund.de}\and}{,}
Katrin Bolsmann\\
\texorpdfstring{\href{mailto:katrin.bolsmann@tu-dortmund.de}{katrin.bolsmann@tu-dortmund.de}}{,}\\
\\
Mario Alex Hollberg\\
\texorpdfstring{\href{mailto:mario-alex.hollberg@tu-dortmund.de}{mario-alex.hollberg@tu-dortmund.de}}{}
}
\date{Abgabe: 22. Mai 2020}
\usepackage{amsmath}
\usepackage{amssymb}
\usepackage{mathtools}
\usepackage[
    math-style=ISO,
    bold-style=ISO,
    sans-style=italic,
    nabla=upright,
    partial=upright,
]{unicode-math}

\setmathfont{Latin Modern Math}

\usepackage[
  locale=DE,
  separate-uncertainty=true,
  per-mode=symbol-or-fraction,
]{siunitx}

\usepackage{multicol}
\setlength{\columnsep}{1pt} %space between columns

\usepackage{booktabs}
\usepackage[x11names, table]{xcolor}
\usepackage{graphicx}
\usepackage{grffile}
\usepackage{xfrac}
\usepackage{xcolor}

\usepackage{float}
\floatplacement{figure}{h}
\floatplacement{table}{h}
\usepackage[
  section,
  below,
]{placeins}

\usepackage{expl3}
\usepackage{xparse}
\ExplSyntaxOn
\NewDocumentCommand \E {} {\symup{e}}
\ExplSyntaxOff

% Literaturverzeichnis
\usepackage[
  backend=biber,
]{biblatex}
% Quellendatenbank
\addbibresource{literatur.bib}

\usepackage[
  version=4,
  math-greek=default,
  text-greek=default,
]{mhchem}

\def\Xint#1{\mathchoice
   {\XXint\displaystyle\textstyle{#1}}%
   {\XXint\textstyle\scriptstyle{#1}}%
   {\XXint\scriptstyle\scriptscriptstyle{#1}}%
   {\XXint\scriptscriptstyle\scriptscriptstyle{#1}}%
   \!\int}
\def\XXint#1#2#3{{\setbox0=\hbox{$#1{#2#3}{\int}$}
     \vcenter{\hbox{$#2#3$}}\kern-.5\wd0}}
\def\ddashint{\Xint=}
\def\dashint{\Xint-}


\raggedcolumns


\begin{document}

\maketitle

\section*{Aufgabe0: Verständnisfragen}

\section*{Aufgabe1: Distributivgesetz in der Floating-Arithmetik}
\begin{equation*}
  \text{zz. }(a \oplus b) \odot c = a \odot c \oplus b \odot c
\end{equation*}
\noindent
Wähle passend zu $t=2$ (Mantissenlänge) und $l=1$ (Exponentenlänge) $a$, $b$ und $c$,
zum Beispiel: $a=\num{0,43e2}$, $b=\num{0,87e1}$ und $c=\num{0,13e3}$

\begin{align*}
\begin{rcases}
  (\num{0,43e2} + \num{0,87e1}) \cdot \num{0,13e3}
  = \num{0,52e2} \cdot \num{0,13e3}
  = \num{0,68e4} \\
  (\num{0,43e2} \cdot \num{0,13e3}) + \num{0,87e1} \cdot \num{0,13e3}
  = \num{0,56e4} \cdot \num{0,11e4}
  = \num{0,67e4}
\end{rcases}
{\ne}
\end{align*}
\noindent
Folglich ist das Distributivgesetz  nicht erfüllt:
$\implies (a \oplus b) \odot c \ne a \odot c \oplus b \odot c$
\section*{Aufgabe2: Rundungsfehler}

\section*{Aufgabe3: Stabilität}
Anfangswertproblem:
\begin{align*}
  \dot{y}(t)&=-y(t), \\ y(0)&=1 \\
  \intertext{Analytische Lösung:} y(t)&=e^{t}
\end{align*}

\subsection*{a)}
Anfangswerte:
\begin{align*}
  y_0&=1\\
  \intertext{und für das symmetrische Euler-Verfahren zusätzlich}
  y_1&=e^{-\Delta t}
\end{align*}
Für eine Schrittweite von $\Delta t = 0,1$ ergibt sich auf dem Intervall [0, 10] folgene Funktion als Lösung des Anfangswertproblems.
\FloatBarrier
\begin{figure}[h]
    \centering
    \includegraphics[width=0.8\textwidth]{plot3a.pdf}
    \caption{Aufgabe 3a: Lösung des AWP mittels Euler-Verfahrens und symmetrischen Euler-Verfahrens und rauschfreien Anfangswerten.}
    \label{fig:plot3a}
\end{figure}
\FloatBarrier
Für die gegebenen Startwerte ist die Lösung des Anfangswertproblems sowohl mittels Euler-Verfahrens als auch mittels symmetrischen Euler Verfahrens sehr genau, da kaum ein Unterschied zwischen den Funktionen erkennbar ist.

\subsection*{b)}
Anfangswerte für das Euler-Verfahren:
\begin{align*}
  y_0&=1-\Delta t\\
  \intertext{Anfangswerte für das symmetrische Euler-Verfahren:}\\
  y_0&=1\\
  y_1&=y_0-\Delta t
\end{align*}
Mit der gleichen Schrittweite wie in a) ergibt sich mit den angegebenen Startwerten folgene Funktion als Lösung des Anfangswertproblems.
\FloatBarrier
\begin{figure}[h]
    \centering
    \includegraphics[width=0.8\textwidth]{plot3b.pdf}
    \caption{Aufgabe 3b: Lösung des AWP mittels Euler Verfahrens und symmetrischen Euler-Verfahrens und rauschbelasteten Anfangswerten.}
    \label{fig:plot3a}
\end{figure}
\FloatBarrier
Hier ist deutlich erkennbar, dass das Rauschen zu einer Instabilität des symmetrischen Euler-Verfahrens führt. Die Lösung dieses Verfahrens weicht auf dem gegebenen Intervall deutlich stärker von der analytischen Lösung ab, als die Lösung, welche durch das symmetrische Euler-Verfahren bestimmt wurde. Damit wurde anhand eines Beispiels gezeigt, dass das symmetrische Euler-Verfahren instabil gegenüber kleinen Fehlern (hier in der Größenordnung der Schrittweite) in den Anfangswerten ist.
\end{document}
